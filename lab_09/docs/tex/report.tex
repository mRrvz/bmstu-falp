\documentclass[12pt]{report}
\usepackage[utf8]{inputenc}
\usepackage[russian]{babel}
%\usepackage[14pt]{extsizes}
\usepackage{listings}
\usepackage{graphicx}
\usepackage{amsmath,amsfonts,amssymb,amsthm,mathtools} 
\usepackage{pgfplots}
\usepackage{filecontents}
\usepackage{float}
\usepackage{indentfirst}
\usepackage{eucal}
\usepackage{enumitem}
%s\documentclass[openany]{book}
\frenchspacing

\usepackage{indentfirst} % Красная строка

\usetikzlibrary{datavisualization}
\usetikzlibrary{datavisualization.formats.functions}

\usepackage{amsmath}


% Для листинга кода:
\lstset{ %
	language=c,                 % выбор языка для подсветки (здесь это С)
	basicstyle=\small\sffamily, % размер и начертание шрифта для подсветки кода
	numbers=left,               % где поставить нумерацию строк (слева\справа)
	numberstyle=\tiny,           % размер шрифта для номеров строк
	stepnumber=1,                   % размер шага между двумя номерами строк
	numbersep=5pt,                % как далеко отстоят номера строк от подсвечиваемого кода
	showspaces=false,            % показывать или нет пробелы специальными отступами
	showstringspaces=false,      % показывать или нет пробелы в строках
	showtabs=false,             % показывать или нет табуляцию в строках
	frame=single,              % рисовать рамку вокруг кода
	tabsize=2,                 % размер табуляции по умолчанию равен 2 пробелам
	captionpos=t,              % позиция заголовка вверху [t] или внизу [b] 
	breaklines=true,           % автоматически переносить строки (да\нет)
	breakatwhitespace=false, % переносить строки только если есть пробел
	escapeinside={\#*}{*)}   % если нужно добавить комментарии в коде
}


\usepackage[left=2cm,right=2cm, top=2cm,bottom=2cm,bindingoffset=0cm]{geometry}
% Для измененных титулов глав:
\usepackage{titlesec, blindtext, color} % подключаем нужные пакеты
\definecolor{gray75}{gray}{0.75} % определяем цвет
\newcommand{\hsp}{\hspace{20pt}} % длина линии в 20pt
% titleformat определяет стиль
\titleformat{\chapter}[hang]{\Huge\bfseries}{\thechapter\hsp\textcolor{gray75}{|}\hsp}{0pt}{\Huge\bfseries}


% plot
\usepackage{pgfplots}
\usepackage{filecontents}
\usetikzlibrary{datavisualization}
\usetikzlibrary{datavisualization.formats.functions}

\begin{document}
	%\def\chaptername{} % убирает "Глава"
	\thispagestyle{empty}
	\begin{titlepage}
		\noindent \begin{minipage}{0.15\textwidth}
			\includegraphics[width=\linewidth]{img/b_logo}
		\end{minipage}
		\noindent\begin{minipage}{0.9\textwidth}\centering
			\textbf{Министерство науки и высшего образования Российской Федерации}\\
			\textbf{Федеральное государственное бюджетное образовательное учреждение высшего образования}\\
			\textbf{~~~«Московский государственный технический университет имени Н.Э.~Баумана}\\
			\textbf{(национальный исследовательский университет)»}\\
			\textbf{(МГТУ им. Н.Э.~Баумана)}
		\end{minipage}
		
		\noindent\rule{18cm}{3pt}
		\newline\newline
		\noindent ФАКУЛЬТЕТ $\underline{\text{«Информатика и системы управления»}}$ \newline\newline
		\noindent КАФЕДРА $\underline{\text{«Программное обеспечение ЭВМ и информационные технологии»}}$\newline\newline\newline\newline\newline
		
		\begin{center}
			\noindent\begin{minipage}{1.1\textwidth}\centering
				\Large\textbf{  Отчет по лабораторной работе №9}\newline
				\textbf{по дисциплине <<Функциональное и логическое}\newline
				\textbf{~~~программирование>>}\newline\newline
			\end{minipage}
		\end{center}
		
		\noindent\textbf{Тема} $\underline{\text{Использование функционалов и рекурсии~~~~~~}}$\newline\newline
		\noindent\textbf{Студент} $\underline{\text{Романов А.В.~~~~~~~~~~~~~~~~~~~~~~~~~~~~~~~~~~~~~~~~}}$\newline\newline
		\noindent\textbf{Группа} $\underline{\text{ИУ7-63Б~~~~~~~~~~~~~~~~~~~~~~~~~~~~~~~~~~~~~~~~~~~~~~~~}}$\newline\newline
		\noindent\textbf{Оценка (баллы)} $\underline{\text{~~~~~~~~~~~~~~~~~~~~~~~~~~~~~~~~~~~~~~~~~~~~~~~}}$\newline\newline
		\noindent\textbf{Преподаватель} $\underline{\text{Толпинская Н.Б., Строганов Ю. В.}}$\newline\newline\newline
		
		\begin{center}
			\vfill
			Москва~---~\the\year
			~г.
		\end{center}
	\end{titlepage}
	
	
\section*{Задание 1}
\subsection*{Постановка задачи}
Написать функцию, которая выбирает из заданного списка только ты числа, которые больше 1 и меньше 10.

\subsection*{Решение}
\begin{lstlisting}
(defun rec-add-to-end (lst elem)
	(cond 
		((cdr lst)
			(rec-add-to-end (cdr lst) elem))
		((listp elem)
			(setf (cdr lst) elem))
		(t (setf (cdr lst) (cons elem Nil))))
	lst)

(defun add-to-end (lst elem)
	(if (null lst)
		(cons elem Nil)
		(rec-add-to-end lst elem)))

(defun select-between (lst)
	(reduce
		#'(lambda (acc el)
			(if (and (> el 1) (< el 10))
				(add-to-end acc el)
				acc))
	lst :initial-value ()))
\end{lstlisting}


\section*{Задание №2}
\subsection*{Постановка задачи}
Написать функцию, вычисляющую декартово произведение двух своих списков-аргументов.

\subsection*{Решение}
\begin{lstlisting}
(defun cartesian-prod (lst1 lst2)
	(mapcan #'(lambda (x1)
		(mapcar #'(lambda (x2)
			(cons x1 x2))
		lst2))
	lst1))
\end{lstlisting}


\section*{Задание №3}
\subsection*{Постановка задачи}
Почему так реализовано \texttt{reduce}, в чем причина? \texttt{(reduce \#'+ ()) -> 0}

\subsection*{Решение}
Функция \texttt{+} --- функционал, который при 0 количестве аргументов возвращает значение 0. Если подать на вход \texttt{reduce} функцию, которая не может обработать 0 аргументов, то вызов \texttt{reduce} с пустым списком в качестве второго аргумента вернет ошибку (\texttt{invalid number of arguments: 0}). При этом, если подано более одного аргумента, то \texttt{reduce} выполняет действия:

\begin{enumerate}
	\item сохраняет первый элемент списка в область памяти;
	\item для всех остальных элементов списка выполняет переданную в качестве первого аргумента функцию, подавая на вход 2 аргумента и сохраняя результат в \texttt{acc}.
\end{enumerate}

\section*{Задание №4}
\subsection*{Постановка задачи}
Пусть \texttt{list-of-lists} --- список, состоящий из списков. Написать функцию, которая вычисляет сумму длин всех элементов \texttt{list-of-lists}, то есть, например, для аргумента \texttt{((1 2) (3 4)) -> 4}

\subsection*{Решение}
\begin{lstlisting}
(defun sum-len-of-list (lst)
	(reduce #'(lambda (acc x) (+ acc (length x))) lst :initial-value 0))
\end{lstlisting}

\section*{Задание №5}
Используя рекурсию, написать функцию, которая по исходному списку стоит список квадратов чисел смешанного структурированного списка

\subsection*{Постановка задачи}
\subsection*{Решение}
\begin{lstlisting}
(defun list-squares-internal (lst acc)
	(if (null lst) acc
		(cond 
			((listp (car lst))
				(add-to-end acc (list-squares-internal (car lst) ())))
			((numberp (car lst))
				(list-squares-internal (cdr lst) (add-to-end acc (* (car lst) (car lst)))))
			(t
				(list-squares-internal (cdr lst) acc)))))

(defun list-squares (lst)
	(list-squares-internal lst ()))
\end{lstlisting}

\section*{Контрольные вопросы}
\textbf{Вопрос 1.} Классификация рекурсивных функций\\

\textbf{Ответ.} \textbf{Рекурсия} --- ссылка на описываемый объект во время его описания.\\

Классификация рекурсивных функций:
\begin{itemize}
	\item простая (рекурсивный вызов --- единственный);
	\item второго порядка (несколько рекурсивных вызовов);
	\item взаимная рекурсия (используются несколько рекурсивных функций, которые могут друг друга вызывать).
	\item хвостовая рекурсия (при очередном вызове рекурсивной функции все действия до входа выполнены, а при выходе ничего более делать не приходится);
	\item дополняемая рекурсия (результат рекурсии используется, как аргумент некоторой другой функции (которую называют \textit{дополняемой функцией}); частный случай --- \texttt{cons}-дополняемая рекурсия).
\end{itemize}
	
\bibliographystyle{utf8gost705u}  % стилевой файл для оформления по ГОСТу
	
\bibliography{51-biblio}          % имя библиографической базы (bib-файла)
	
	
\end{document}
