\documentclass[12pt]{report}
\usepackage[utf8]{inputenc}
\usepackage[russian]{babel}
%\usepackage[14pt]{extsizes}
\usepackage{listings}
\usepackage{graphicx}
\usepackage{amsmath,amsfonts,amssymb,amsthm,mathtools} 
\usepackage{pgfplots}
\usepackage{filecontents}
\usepackage{float}
\usepackage{indentfirst}
\usepackage{eucal}
\usepackage{enumitem}
%s\documentclass[openany]{book}
\frenchspacing

\usepackage{indentfirst} % Красная строка

\usetikzlibrary{datavisualization}
\usetikzlibrary{datavisualization.formats.functions}

\usepackage{amsmath}


% Для листинга кода:
\lstset{ %
	language=c,                 % выбор языка для подсветки (здесь это С)
	basicstyle=\small\sffamily, % размер и начертание шрифта для подсветки кода
	numbers=left,               % где поставить нумерацию строк (слева\справа)
	numberstyle=\tiny,           % размер шрифта для номеров строк
	stepnumber=1,                   % размер шага между двумя номерами строк
	numbersep=5pt,                % как далеко отстоят номера строк от подсвечиваемого кода
	showspaces=false,            % показывать или нет пробелы специальными отступами
	showstringspaces=false,      % показывать или нет пробелы в строках
	showtabs=false,             % показывать или нет табуляцию в строках
	frame=single,              % рисовать рамку вокруг кода
	tabsize=2,                 % размер табуляции по умолчанию равен 2 пробелам
	captionpos=t,              % позиция заголовка вверху [t] или внизу [b] 
	breaklines=true,           % автоматически переносить строки (да\нет)
	breakatwhitespace=false, % переносить строки только если есть пробел
	escapeinside={\#*}{*)}   % если нужно добавить комментарии в коде
}


\usepackage[left=2cm,right=2cm, top=2cm,bottom=2cm,bindingoffset=0cm]{geometry}
% Для измененных титулов глав:
\usepackage{titlesec, blindtext, color} % подключаем нужные пакеты
\definecolor{gray75}{gray}{0.75} % определяем цвет
\newcommand{\hsp}{\hspace{20pt}} % длина линии в 20pt
% titleformat определяет стиль
\titleformat{\chapter}[hang]{\Huge\bfseries}{\thechapter\hsp\textcolor{gray75}{|}\hsp}{0pt}{\Huge\bfseries}


% plot
\usepackage{pgfplots}
\usepackage{filecontents}
\usetikzlibrary{datavisualization}
\usetikzlibrary{datavisualization.formats.functions}

\begin{document}
	%\def\chaptername{} % убирает "Глава"
	\thispagestyle{empty}
	\begin{titlepage}
		\noindent \begin{minipage}{0.15\textwidth}
			\includegraphics[width=\linewidth]{img/b_logo}
		\end{minipage}
		\noindent\begin{minipage}{0.9\textwidth}\centering
			\textbf{Министерство науки и высшего образования Российской Федерации}\\
			\textbf{Федеральное государственное бюджетное образовательное учреждение высшего образования}\\
			\textbf{~~~«Московский государственный технический университет имени Н.Э.~Баумана}\\
			\textbf{(национальный исследовательский университет)»}\\
			\textbf{(МГТУ им. Н.Э.~Баумана)}
		\end{minipage}
		
		\noindent\rule{18cm}{3pt}
		\newline\newline
		\noindent ФАКУЛЬТЕТ $\underline{\text{«Информатика и системы управления»}}$ \newline\newline
		\noindent КАФЕДРА $\underline{\text{«Программное обеспечение ЭВМ и информационные технологии»}}$\newline\newline\newline\newline\newline
		
		\begin{center}
			\noindent\begin{minipage}{1.1\textwidth}\centering
				\Large\textbf{  Отчет по лабораторной работе №2}\newline
				\textbf{по дисциплине <<Функциональное и логическое}\newline
				\textbf{~~~программирование>>}\newline\newline
			\end{minipage}
		\end{center}
		
		\noindent\textbf{Тема} $\underline{\text{Функции языка Lisp~~~~~~~~~~~}}$\newline\newline
		\noindent\textbf{Студент} $\underline{\text{Романов А.В.~~~~~~~~~~~~~~~}}$\newline\newline
		\noindent\textbf{Группа} $\underline{\text{ИУ7-63Б~~~~~~~~~~~~~~~~~~~~~~~}}$\newline\newline
		\noindent\textbf{Оценка (баллы)} $\underline{\text{~~~~~~~~~~~~~~~~~~~~~~}}$\newline\newline
		\noindent\textbf{Преподаватель} $\underline{\text{Толпинская Н.Б.}}$\newline\newline\newline
		
		\begin{center}
			\vfill
			Москва~---~\the\year
			~г.
		\end{center}
	\end{titlepage}
	
	
	
\section*{Задание 1}
\subsection*{Постановка задачи}
	
Используя только функции CAR и CDR, написать
выражения, возвращающие второй, третий, четвертый элементы заданного
списка.
	
\subsection*{Решение}

\begin{lstlisting}[label=first,caption=Решение задания №1, language=lisp]
(car (cdr `(fst snd thd etc)))
(car (cdr (cdr `(fst snd thd etc))))
(car (cdr (cdr (cdr `(fst snd thd etc)))))
\end{lstlisting}

\section*{Задание №2}
\subsection*{Постановка задачи}
Что будет в результате вычисления выражений?
\subsection*{Решение}

\begin{lstlisting}[label=second,caption=Решение задания №2, language=lisp]
(caadr `((blue cube) (red pyaramid))) ; red
(cdar `((abc) (def) (ghi))) ; Nil
(cadr `((abc) (def) (ghi))) ; (def)
(caddr `((abc) (def) (ghi))) ; (ghi)
\end{lstlisting}

\section*{Задание №3}
\subsection*{Постановка задачи}
Напишите результат вычисления выражений:
\subsection*{Решение}
\begin{lstlisting}[label=second,caption=Решение задания №3, language=lisp]
(list `Fred `and  Wilma) ; Wilma is unbound. (list `Fred `and `Wilma) -> (Fred and Wilma)
(list `Fred `(and Wilma)) ; (Fred (and Wilma))
(cons Nil Nil) ; (Nil)
(cons T Nil) ; (T)
\end{lstlisting}

\begin{lstlisting}
(cons Nil T) ; (Nil . T)
(list Nil) ; (Nil)
(cons (T) Nil) ; undefined function. (cons `(T) Nil) -> ((T))
(list `(one two) `(free temp)) ; ((one two) (free temp))
(cons `Fred `(and Wilma)) ; (Fred and Wilma)
(cons `Fred `(Wilma)) ; (Fred Wilma)
(list Nil Nil) ; (Nil Nil)
(list T Nil) ; (T Nil)
(list Nil T) ; (Nil T)
(cons T (list Nil)) ; (T Nil)
(list (T) Nil) ; undefined function. (list `(T) Nil) -> ((T) Nil)
(cons `(one two) `(free temp)) ; ((one two) free temp)
\end{lstlisting}

\section*{Задание №4}
\subsection*{Постановка задачи}
Написать функцию (f ar1 ar2 ar3 ar4), возвращаущую список: ((ar1 ar) (ar3 ar4)).""\newline
\indent Написать функцию (f ar1 ar2), возвращаущую ((ar1) (ar2)).""\newline
\indent Написать функцию (f ar1), возвращаущую (((ar1))).""\newline
\indent Представить результаты в виде списочных ячеек.
\subsection*{Решение}
\begin{lstlisting}[label=third,caption=Решение задания №4 (функция №1), language=lisp]
(defun f (ar1 ar2 ar3 ar4)
(cons (cons ar1 (cons ar2 Nil)) (cons (cons ar3 (cons ar4 Nil)) Nil)))
\end{lstlisting}

\begin{lstlisting}[label=third,caption=Решение задания №4 (функция №2), language=lisp]
(defun f (ar1 ar2)
(cons (cons ar1 Nil) (cons (cons ar2 Nil) Nil))
\end{lstlisting}

\begin{lstlisting}[label=third,caption=Решение задания №4 (функция №3), language=lisp]
(defun f (ar1)
(cons (cons (cons ar1 Nil) Nil) Nil)

\end{lstlisting}

	
\section*{Контрольные вопросы}
\textbf{Вопрос 1.} Классификация функций языка Lisp.
	
\textbf{Ответ.} 
	
\begin{itemize}
	\item чистые (с фиксированным количеством аргументов) математические

	функции;

	\item рекурсивные функции;

	\item специальные функции – формы (принимают произвольное количе-

	ство аргументов или по разному обрабатывают аргументы);

	\item псевдофункции (создающие «эффект» – отображающие на экране

	процесс обработки данных и т.п.);

	\item функции с вариативными значениями, выбирающие одно значение;

	\item функции высших порядков – функционалы (используются для по-

	строения синтаксически управляемых программ);
\end{itemize}

\textbf{Вопрос 2.} Базис языка Lisp. \newline
\indent\textbf{Ответ. }
Базис языка представлен:
\begin{itemize}
	\item структурами и атомами;
	\item функциями;
\end{itemize}

Функции, входящие в базис языка:
\begin{itemize}
	\item atom, eq, cons, car, cdr;

	\item cond, quote, lambda, eval, label.
\end{itemize}
	
\textbf{Вопрос 3.} Функции \textbf{car}, \textbf{cdr}.
	
\textbf{Ответ.} Функции $car$, $cdr$ являются базовыми функциями доступа к
данным. car принимает точечную пару или список в качестве аргумента
и возвращает первый элемент или $Nil$, соответственно. $cdr$ принимает точечную пару или список в качестве аргумента и возвращает все элементы
кроме первого или $Nil$, соответственно.""\newline
	
\textbf{Вопрос 4.} Функции \textbf{list}, \textbf{cons}.
	
\textbf{Ответ.} Функции $list$, $cons$ являются функциями создания списков
($cons$ – базовая, $list$ – нет). $cons$ создает списочную ячейку и устанавливает два указателя на аргументы. $list$ принимает переменное число аргументов и возвращает список, элементы которого – переданные в функцию
аргументы.
	
\bibliographystyle{utf8gost705u}  % стилевой файл для оформления по ГОСТу
	
\bibliography{51-biblio}          % имя библиографической базы (bib-файла)
	
	
\end{document}
