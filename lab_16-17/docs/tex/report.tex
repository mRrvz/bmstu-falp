\documentclass[12pt]{report}
\usepackage[utf8]{inputenc}
\usepackage[russian]{babel}
%\usepackage[14pt]{extsizes}
\usepackage{listings}
\usepackage{graphicx}
\usepackage{amsmath,amsfonts,amssymb,amsthm,mathtools} 
\usepackage{pgfplots}
\usepackage{filecontents}
\usepackage{float}
\usepackage{comment}
\usepackage{indentfirst}
\usepackage{eucal}
\usepackage{enumitem}
%s\documentclass[openany]{book}
\frenchspacing

\usepackage{indentfirst} % Красная строка

\usetikzlibrary{datavisualization}
\usetikzlibrary{datavisualization.formats.functions}

\usepackage{amsmath}


% Для листинга кода:
\lstset{ %
	language=c,                 % выбор языка для подсветки (здесь это С)
	basicstyle=\small\sffamily, % размер и начертание шрифта для подсветки кода
	numbers=left,               % где поставить нумерацию строк (слева\справа)
	numberstyle=\tiny,           % размер шрифта для номеров строк
	stepnumber=1,                   % размер шага между двумя номерами строк
	numbersep=5pt,                % как далеко отстоят номера строк от подсвечиваемого кода
	showspaces=false,            % показывать или нет пробелы специальными отступами
	showstringspaces=false,      % показывать или нет пробелы в строках
	showtabs=false,             % показывать или нет табуляцию в строках
	frame=single,              % рисовать рамку вокруг кода
	tabsize=2,                 % размер табуляции по умолчанию равен 2 пробелам
	captionpos=t,              % позиция заголовка вверху [t] или внизу [b] 
	breaklines=true,           % автоматически переносить строки (да\нет)
	breakatwhitespace=false, % переносить строки только если есть пробел
	escapeinside={\#*}{*)}   % если нужно добавить комментарии в коде
}


\usepackage[left=2cm,right=2cm, top=2cm,bottom=2cm,bindingoffset=0cm]{geometry}
% Для измененных титулов глав:
\usepackage{titlesec, blindtext, color} % подключаем нужные пакеты
\definecolor{gray75}{gray}{0.75} % определяем цвет
\newcommand{\hsp}{\hspace{20pt}} % длина линии в 20pt
% titleformat определяет стиль
\titleformat{\chapter}[hang]{\Huge\bfseries}{\thechapter\hsp\textcolor{gray75}{|}\hsp}{0pt}{\Huge\bfseries}


% plot
\usepackage{pgfplots}
\usepackage{filecontents}
\usetikzlibrary{datavisualization}
\usetikzlibrary{datavisualization.formats.functions}

\begin{document}
	%\def\chaptername{} % убирает "Глава"
	\thispagestyle{empty}
	\begin{titlepage}
		\noindent \begin{minipage}{0.15\textwidth}
			\includegraphics[width=\linewidth]{img/b_logo}
		\end{minipage}
		\noindent\begin{minipage}{0.9\textwidth}\centering
			\textbf{Министерство науки и высшего образования Российской Федерации}\\
			\textbf{Федеральное государственное бюджетное образовательное учреждение высшего образования}\\
			\textbf{~~~«Московский государственный технический университет имени Н.Э.~Баумана}\\
			\textbf{(национальный исследовательский университет)»}\\
			\textbf{(МГТУ им. Н.Э.~Баумана)}
		\end{minipage}
		
		\noindent\rule{18cm}{3pt}
		\newline\newline
		\noindent ФАКУЛЬТЕТ $\underline{\text{«Информатика и системы управления»}}$ \newline\newline
		\noindent КАФЕДРА $\underline{\text{«Программное обеспечение ЭВМ и информационные технологии»}}$\newline\newline\newline\newline\newline
		
		\begin{center}
			\noindent\begin{minipage}{1.1\textwidth}\centering
				\Large\textbf{  Отчет по лабораторной работе №16 - 17}\newline
				\textbf{по дисциплине <<Функциональное и логическое}\newline
				\textbf{~~~программирование>>}\newline\newline
			\end{minipage}
		\end{center}
		
		\noindent\textbf{Тема} $\underline{\text{Использование правил в программе на Prolog}}$\newline\newline
		\noindent\textbf{Студент} $\underline{\text{Романов А.В.~~~~~~~~~~~~~~~~~~~~~~~~~~~~~~~~~~~~~~~}}$\newline\newline
		\noindent\textbf{Группа} $\underline{\text{ИУ7-63Б~~~~~~~~~~~~~~~~~~~~~~~~~~~~~~~~~~~~~~~~~~~~~~~}}$\newline\newline
		\noindent\textbf{Оценка (баллы)} $\underline{\text{~~~~~~~~~~~~~~~~~~~~~~~~~~~~~~~~~~~~~~~~~~~~~~}}$\newline\newline
		\noindent\textbf{Преподаватель} $\underline{\text{Толпинская Н.Б., Строганов Ю. В.}}$\newline\newline\newline
		
		\begin{center}
			\vfill
			Москва~---~\the\year
			~г.
		\end{center}
	\end{titlepage}
	

\chapter*{Лабораторная работа №16}
\section*{Постановка задачи}

Создать базу знаний: <<ПРЕДКИ>>, позволяющую наиболее эффективным способом (за меньшее количество шагов, что обеспечивается меньшим количеством предложений БЗ – правил), и используя разные варианты (примеры) одного вопроса, определить (указать: какой вопрос для какого варианта):

\begin{enumerate}
	\item По имени субъекта определить всех его бабушек (предки 2-го колена);
	\item По имени субъекта определить всех его дедушек (предки 2-го колена);
	\item По имени субъекта определить всех его бабушек и дедушек (предки 2-го колена);
	\item По имени субъекта определить его бабушку по материнской линии (предки 2-го колена);
	\item По имени субъекта определить его бабушку и дедушку по материнской линии (предки 2-го колена).
\end{enumerate}

Минимизировать количество правил и количество вариантов вопросов. Использовать конъюнктивные правила и простой вопрос.

Для одного из вариантов ВОПРОСА и конкретной БЗ составить таблицу, отражающую конкретный порядок работы системы, с объяснениями:

\begin{itemize}
	\item очередная проблема на каждом шаге и метод ее решения,
	\item каково новое текущее состояние резольвенты, как получено,
	\item какие дальнейшие действия? (запускается ли алгоритм унификации? Каких термов? Почему этих?),
	\item вывод по результатам очередного шага и дальнейшие действия.
\end{itemize}

Так как резольвента хранится в виде стека, то состояние резольвенты требуется отображать в столбик: вершина – сверху! Новый шаг надо начинать с нового состояния резольвенты!

\subsection*{Решение}
\begin{lstlisting}
domains
	sex = symbol
	name = string
	man = man(sex, name)

predicates
	parent(man, man)
	grandparent(man, sex, name)
	
clauses
	grandparent(man(Sex, GrandPName), PSex, Name) :-
	parent(man(Sex, GrandPName), man(PSex, PName)),
	parent(man(PSex, PName), man(_, Name)).
	
	parent(man(f, "Natalia"), man(m, "Alexey")).
	parent(man(m, "Vasiliy"), man(m, "Alexey")).
	parent(man(f, "Galya"), man(f, "Natalia")).
	parent(man(m, "Sergey"), man(f, "Natalia")).
	parent(man(f, "Lyuda"), man(m,"Vasiliy")).
	parent(man(m, "Vasiliy"), man(m, "Vasiliy")).

goal
	%grandparent(man(f, GrandPName), _, "Alexey").
	%grandparent(man(m, GrandPName), _, "Alexey").
	%grandparent(man(_, GrandPName), _, "Alexey").
	%grandparent(man(f, GrandPName), f, "Alexey").
	grandparent(man(_, GrandPName), f, "Alexey").
\end{lstlisting}

\chapter*{Лабораторная работа №17}
\section*{Постановка задачи}

В одной программе написать правила, позволяющие найти:

\begin{enumerate}
	\item Максимум из двух чисел:
	\begin{itemize}
		\item Без использования отсечения;
		\item С использованием отсечения;
	\end{itemize}
	\item Максимум из трех чисел:
	\begin{itemize}
		\item Без использования отсечения;
		\item С использованием отсечения.
	\end{itemize}
\end{enumerate}

Убедиться в правильности результатов. Для каждого случая из пункта 2 обосновать необходимость всех условий тела. Для одного из вариантов ВОПРОСА и каждого варианта задания 2 составить таблицу, отражающую конкретный порядок работы системы.

Так как резольвента хранится в виде стека, то состояние резольвенты требуется отображать в столбик: вершина – сверху! Новый шаг надо начинать с нового состояния резольвенты!

Требуется ответить на вопрос: <<За счет чего может быть достигнута эффективность работы системы?>>


\subsection*{Решение}
\begin{lstlisting}
domains
	num = integer

predicates
	max2(num, num, num)
	max3(num, num, num, num)
	
	max2clipping(num, num, num)
	max3clipping(num, num, num, num)

clauses
	max2(N1, N2, N2) :- N2 >= N1.
	max2(N1, N2, N1) :- N1 >= N2.
	
	max3(N1, N2, N3, N3) :- N3 >= N1, N3 >= N2.
	max3(N1, N2, N3, N2) :- N2 >= N1, N2 >= N3.
	max3(N1, N2, N3, N1) :- N1 >= N2, N1 >= N3.
	
	max2clipping(N1, N2, N2) :- N2 >= N1, !.
	max2clipping(N1, _, N1).
	
	max3clipping(N1, N2, N3, N3) :- N3 >= N2, N3 >= N1, !.
	max3clipping(N1, N2, _, N1) :- N1 >= N2, !.
	max3clipping(_, N2, _, N2).

goal
	%max2clipping(1, 4, Max).
	max3(133, 4, 5, Max).
\end{lstlisting}


\bibliographystyle{utf8gost705u}  % стилевой файл для оформления по ГОСТу
\bibliography{51-biblio}          % имя библиографической базы (bib-файла)
	
\end{document}
