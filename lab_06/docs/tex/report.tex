\documentclass[12pt]{report}
\usepackage[utf8]{inputenc}
\usepackage[russian]{babel}
%\usepackage[14pt]{extsizes}
\usepackage{listings}
\usepackage{graphicx}
\usepackage{amsmath,amsfonts,amssymb,amsthm,mathtools} 
\usepackage{pgfplots}
\usepackage{filecontents}
\usepackage{float}
\usepackage{indentfirst}
\usepackage{eucal}
\usepackage{enumitem}
%s\documentclass[openany]{book}
\frenchspacing

\usepackage{indentfirst} % Красная строка

\usetikzlibrary{datavisualization}
\usetikzlibrary{datavisualization.formats.functions}

\usepackage{amsmath}


% Для листинга кода:
\lstset{ %
	language=c,                 % выбор языка для подсветки (здесь это С)
	basicstyle=\small\sffamily, % размер и начертание шрифта для подсветки кода
	numbers=left,               % где поставить нумерацию строк (слева\справа)
	numberstyle=\tiny,           % размер шрифта для номеров строк
	stepnumber=1,                   % размер шага между двумя номерами строк
	numbersep=5pt,                % как далеко отстоят номера строк от подсвечиваемого кода
	showspaces=false,            % показывать или нет пробелы специальными отступами
	showstringspaces=false,      % показывать или нет пробелы в строках
	showtabs=false,             % показывать или нет табуляцию в строках
	frame=single,              % рисовать рамку вокруг кода
	tabsize=2,                 % размер табуляции по умолчанию равен 2 пробелам
	captionpos=t,              % позиция заголовка вверху [t] или внизу [b] 
	breaklines=true,           % автоматически переносить строки (да\нет)
	breakatwhitespace=false, % переносить строки только если есть пробел
	escapeinside={\#*}{*)}   % если нужно добавить комментарии в коде
}


\usepackage[left=2cm,right=2cm, top=2cm,bottom=2cm,bindingoffset=0cm]{geometry}
% Для измененных титулов глав:
\usepackage{titlesec, blindtext, color} % подключаем нужные пакеты
\definecolor{gray75}{gray}{0.75} % определяем цвет
\newcommand{\hsp}{\hspace{20pt}} % длина линии в 20pt
% titleformat определяет стиль
\titleformat{\chapter}[hang]{\Huge\bfseries}{\thechapter\hsp\textcolor{gray75}{|}\hsp}{0pt}{\Huge\bfseries}


% plot
\usepackage{pgfplots}
\usepackage{filecontents}
\usetikzlibrary{datavisualization}
\usetikzlibrary{datavisualization.formats.functions}

\begin{document}
	%\def\chaptername{} % убирает "Глава"
	\thispagestyle{empty}
	\begin{titlepage}
		\noindent \begin{minipage}{0.15\textwidth}
			\includegraphics[width=\linewidth]{img/b_logo}
		\end{minipage}
		\noindent\begin{minipage}{0.9\textwidth}\centering
			\textbf{Министерство науки и высшего образования Российской Федерации}\\
			\textbf{Федеральное государственное бюджетное образовательное учреждение высшего образования}\\
			\textbf{~~~«Московский государственный технический университет имени Н.Э.~Баумана}\\
			\textbf{(национальный исследовательский университет)»}\\
			\textbf{(МГТУ им. Н.Э.~Баумана)}
		\end{minipage}
		
		\noindent\rule{18cm}{3pt}
		\newline\newline
		\noindent ФАКУЛЬТЕТ $\underline{\text{«Информатика и системы управления»}}$ \newline\newline
		\noindent КАФЕДРА $\underline{\text{«Программное обеспечение ЭВМ и информационные технологии»}}$\newline\newline\newline\newline\newline
		
		\begin{center}
			\noindent\begin{minipage}{1.1\textwidth}\centering
				\Large\textbf{  Отчет по лабораторной работе №6}\newline
				\textbf{по дисциплине <<Функциональное и логическое}\newline
				\textbf{~~~программирование>>}\newline\newline
			\end{minipage}
		\end{center}
		
		\noindent\textbf{Тема} $\underline{\text{Использование управляющих структур, работа со списками}}$\newline\newline
		\noindent\textbf{Студент} $\underline{\text{Романов А.В.~~~~~~~~~~~~~~~~~~~~~~~~~~~~~~~~~~~~~~~~~~~~~~~~~~~~~~~~~~~~}}$\newline\newline
		\noindent\textbf{Группа} $\underline{\text{ИУ7-63Б~~~~~~~~~~~~~~~~~~~~~~~~~~~~~~~~~~~~~~~~~~~~~~~~~~~~~~~~~~~~~~~~~~~~}}$\newline\newline
		\noindent\textbf{Оценка (баллы)} $\underline{\text{~~~~~~~~~~~~~~~~~~~~~~~~~~~~~~~~~~~~~~~~~~~~~~~~~~~~~~~~~~~~~~~~~~~}}$\newline\newline
		\noindent\textbf{Преподаватель} $\underline{\text{Толпинская Н.Б., Строганов Ю. В.~~~~~~~~~~~~~~~~~~~~}}$\newline\newline\newline
		
		\begin{center}
			\vfill
			Москва~---~\the\year
			~г.
		\end{center}
	\end{titlepage}
	
	
\section*{Задание 1}
\subsection*{Постановка задачи}
Чем принципиально отличаются функции \texttt{cons}, \texttt{list}, \texttt{append}?\\
\indent Пусть \texttt{(setf lst1 '(a b))} \texttt{(setf lst2 '(c d))}\\
\indent Каковы результаты следующих выражений?

\begin{lstlisting}
(cons lst1 lst2)
(list lst1 lst2)
(append lst1 lst2)
\end{lstlisting}

\subsection*{Решение}
\begin{enumerate}
	\item \texttt{((A B) C D)}
	\item \texttt{((A B) (C D))}
	\item \texttt{(A B C D)}
\end{enumerate}

\section*{Задание №2}
\subsection*{Постановка задачи}
Каковы результаты вычисления следующих выражений?

\begin{lstlisting}
(reverse ())
(last ())
(reverse '(a))
(last '(a))
(reverse '((a b c)))
(last '((a b c)))
\end{lstlisting}


\subsection*{Решение}
\begin{enumerate}
	\item \texttt{Nil}
	\item \texttt{Nil}
	\item \texttt{(a)}
	\item \texttt{(a)}
	\item \texttt{((a b c))}
	\item \texttt{((a b c))}
\end{enumerate}

\section*{Задание №3}
\subsection*{Постановка задачи}
Написать, по крайней мере, два варианта функции, которая возвращает последний элемент своего списка-аргумента

\subsection*{Решение}
\begin{lstlisting}[label=third,caption=Решение задания №3, language=lisp]
(defun last (lst)
	(if (cdr lst)
		(last-elem (cdr lst))
		(car lst))
)

(defun last-reduce (lst)
	(reduce #'(lambda (acc current) current) lst)
)
\end{lstlisting}

\section*{Задание №4}
\subsection*{Постановка задачи}
Написать, по крайней мере, два варианта функции, которая возвращает свой список-аргумент без последнего элемента

\subsection*{Решение}
\begin{lstlisting}[label=third,caption=Решение задания №3, language=lisp]
(defun init (lst)
	(defun wrapper (lst acc)
		(if (cdr lst)
			(wrapper (cdr lst) (append acc (cons (car lst) Nil)))
			acc)
		)
	(wrapper lst ())
)

(defun init-reverse (lst)
	(reverse (cdr (reverse lst)))
)
\end{lstlisting}

\section*{Задание №5}
\subsection*{Постановка задачи}
Написать простой вариант игры в кости, в котором бросаются две правильные кости. Если сумма выпавших очков равна 7 или 11 --- выигрыш, если выпало $(1, 1)$ или $(6, 6)$ --- игрок получает право снова бросить кости, во всех остальных случаях ход переходит ко второму игроку, но запоминается сумма выпавших очков. Если второй игрок не выигрывает абсолютно, то выигрывает тот игрок, у которого больше очков. Результат игры и значения выпавших костей выводить на экран с помощью функции \texttt{print}.

\subsection*{Решение}
\begin{lstlisting}[label=5,caption=Решение задания №5, language=lisp]
(defun random-cube-value ()
	(list (random 7) (random 7)))

(defun dices-sum (pair)
	(+ (car pair) (car (cdr pair))))

(defun check-absolute-win (sum)
	(or (= sum 7) (= sum 11)))

(defun check-rerun (pair)
	(let* ((fst (car pair))
		   (snd (car (cdr pair))))
	(or (= fst snd 1) (= fst snd 6)))
)

(defun dices ()
	(let* ((fst-dices-pair (random-cube-value))
			(fst-dices-sum (dices-sum fst-dices-pair)))
				(format T "Player one dices: ~a.~%" fst-dices-pair)
				(cond ((check-absolute-win fst-dices-sum)
					(format T "Player one win!~%"))
				((check-rerun fst-dices-pair)
					(format T "Rerun!~%") (dices))
				(t (let* ((snd-dices-pair (random-cube-value))
					   	    (snd-dices-sum (dices-sum snd-dices-pair)))
					          (format T "Player two dices: ~a.~%" snd-dices-pair)
						  	  (cond ((check-absolute-win snd-dices-sum)
						      	(format T "Player two win!~%"))
						  		((> fst-dices-sum snd-dices-sum)
							  		(format T "Player one win!~%"))
						  		(t
							  		(format T "Player two win!~%")))))))
)

\end{lstlisting}


\section*{Контрольные вопросы}
\textbf{Вопрос 1.} Структуроразрушающие и не разрушающие структуру списка функции\newline
\indent\textbf{Ответ.} \textbf{Не разрушающие структуру списка функции.} Данные функции не меняют сам объект-аргумент, а создают копию. К таким функциям относятся: 
\texttt{append, reverse, last, nth, nthcdr, length, remove, subst} и прочие.\\

\indent \textbf{Структуроразрушающие функции.} Данные функции меняют сам объект-аргумент, невозможно вернуться к исходному списку. Чаще всего такие функции начинаются с префикса \texttt{n-}. К такми функция относятся: \texttt{nreverse, nconc, nsubst} и прочие.\\

\textbf{Вопрос 2.} Отличие в работе функций \texttt{cons}, \texttt{list}, \texttt{append} и в их результате. \newline
\indent\textbf{Ответ. }Функция \texttt{cons} --- чисто математическая, конструирует списковую ячейку, которая может вовсе и не быть списком. Является списком только в том случае, если вторым аргументом передан список.\\

Функция \texttt{list} --- форма, принимает произвольное количество аргументов и конструирует из них список. Результат --- всегда список. При нуле аргументов возвращает пустой список.\\

Функция \texttt{append} --- форма, принимает на вход произвольное количество аргументов и для всех аргументов, кроме последнего, создает копию, ссылая при этом последний элемент каждого списка-аргумента на первый элемент следующего по порядку списка-аргумента. Копирование для последнего не делается в целях эффективности.\\
	
\bibliographystyle{utf8gost705u}  % стилевой файл для оформления по ГОСТу
	
\bibliography{51-biblio}          % имя библиографической базы (bib-файла)
	
	
\end{document}
