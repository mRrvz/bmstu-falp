\documentclass[12pt]{report}
\usepackage[utf8]{inputenc}
\usepackage[russian]{babel}
%\usepackage[14pt]{extsizes}
\usepackage{listings}
\usepackage{graphicx}
\usepackage{amsmath,amsfonts,amssymb,amsthm,mathtools} 
\usepackage{pgfplots}
\usepackage{filecontents}
\usepackage{float}
\usepackage{indentfirst}
\usepackage{eucal}
\usepackage{enumitem}
%s\documentclass[openany]{book}
\frenchspacing

\usepackage{indentfirst} % Красная строка

\usetikzlibrary{datavisualization}
\usetikzlibrary{datavisualization.formats.functions}

\usepackage{amsmath}


% Для листинга кода:
\lstset{ %
	language=c,                 % выбор языка для подсветки (здесь это С)
	basicstyle=\small\sffamily, % размер и начертание шрифта для подсветки кода
	numbers=left,               % где поставить нумерацию строк (слева\справа)
	numberstyle=\tiny,           % размер шрифта для номеров строк
	stepnumber=1,                   % размер шага между двумя номерами строк
	numbersep=5pt,                % как далеко отстоят номера строк от подсвечиваемого кода
	showspaces=false,            % показывать или нет пробелы специальными отступами
	showstringspaces=false,      % показывать или нет пробелы в строках
	showtabs=false,             % показывать или нет табуляцию в строках
	frame=single,              % рисовать рамку вокруг кода
	tabsize=2,                 % размер табуляции по умолчанию равен 2 пробелам
	captionpos=t,              % позиция заголовка вверху [t] или внизу [b] 
	breaklines=true,           % автоматически переносить строки (да\нет)
	breakatwhitespace=false, % переносить строки только если есть пробел
	escapeinside={\#*}{*)}   % если нужно добавить комментарии в коде
}


\usepackage[left=2cm,right=2cm, top=2cm,bottom=2cm,bindingoffset=0cm]{geometry}
% Для измененных титулов глав:
\usepackage{titlesec, blindtext, color} % подключаем нужные пакеты
\definecolor{gray75}{gray}{0.75} % определяем цвет
\newcommand{\hsp}{\hspace{20pt}} % длина линии в 20pt
% titleformat определяет стиль
\titleformat{\chapter}[hang]{\Huge\bfseries}{\thechapter\hsp\textcolor{gray75}{|}\hsp}{0pt}{\Huge\bfseries}


% plot
\usepackage{pgfplots}
\usepackage{filecontents}
\usetikzlibrary{datavisualization}
\usetikzlibrary{datavisualization.formats.functions}

\begin{document}
	%\def\chaptername{} % убирает "Глава"
	\thispagestyle{empty}
	\begin{titlepage}
		\noindent \begin{minipage}{0.15\textwidth}
			\includegraphics[width=\linewidth]{img/b_logo}
		\end{minipage}
		\noindent\begin{minipage}{0.9\textwidth}\centering
			\textbf{Министерство науки и высшего образования Российской Федерации}\\
			\textbf{Федеральное государственное бюджетное образовательное учреждение высшего образования}\\
			\textbf{~~~«Московский государственный технический университет имени Н.Э.~Баумана}\\
			\textbf{(национальный исследовательский университет)»}\\
			\textbf{(МГТУ им. Н.Э.~Баумана)}
		\end{minipage}
		
		\noindent\rule{18cm}{3pt}
		\newline\newline
		\noindent ФАКУЛЬТЕТ $\underline{\text{«Информатика и системы управления»}}$ \newline\newline
		\noindent КАФЕДРА $\underline{\text{«Программное обеспечение ЭВМ и информационные технологии»}}$\newline\newline\newline\newline\newline
		
		\begin{center}
			\noindent\begin{minipage}{1.1\textwidth}\centering
				\Large\textbf{  Отчет по лабораторной работе №7}\newline
				\textbf{по дисциплине <<Функциональное и логическое}\newline
				\textbf{~~~программирование>>}\newline\newline
			\end{minipage}
		\end{center}
		
		\noindent\textbf{Тема} $\underline{\text{Использование управляющих структур, работа со списками}}$\newline\newline
		\noindent\textbf{Студент} $\underline{\text{Романов А.В.~~~~~~~~~~~~~~~~~~~~~~~~~~~~~~~~~~~~~~~~~~~~~~~~~~~~~~~~~~~~}}$\newline\newline
		\noindent\textbf{Группа} $\underline{\text{ИУ7-63Б~~~~~~~~~~~~~~~~~~~~~~~~~~~~~~~~~~~~~~~~~~~~~~~~~~~~~~~~~~~~~~~~~~~~}}$\newline\newline
		\noindent\textbf{Оценка (баллы)} $\underline{\text{~~~~~~~~~~~~~~~~~~~~~~~~~~~~~~~~~~~~~~~~~~~~~~~~~~~~~~~~~~~~~~~~~~~}}$\newline\newline
		\noindent\textbf{Преподаватель} $\underline{\text{Толпинская Н.Б., Строганов Ю. В.~~~~~~~~~~~~~~~~~~~~}}$\newline\newline\newline
		
		\begin{center}
			\vfill
			Москва~---~\the\year
			~г.
		\end{center}
	\end{titlepage}
	
	
\section*{Задание 1}
\subsection*{Постановка задачи}
Написать функцию, которая по своему аргументу-списку \texttt{lst} определяет, является ли он полиндромом (то есть равны ли \texttt{lst} и \texttt{(reverse lst)})

\subsection*{Решение}
\begin{lstlisting}
(defun is-palindrome (lst)
	(equal lst (reverse lst)))
\end{lstlisting}

\section*{Задание №2}
\subsection*{Постановка задачи}
Написать предикат \texttt{set-equal}, который возвращает \texttt{t}, если два его множества-аргумента содержат одни и те же элементы, порядок которых не имеет значения

\subsection*{Решение}
\begin{lstlisting}
(defun set-equal (set1 set2)
	(not (set-difference set1 set2)))
\end{lstlisting}


\section*{Задание №3}
\subsection*{Постановка задачи}
Напишите необходимые функции, которые обрабатывают таблицу из точечных пар: \texttt{(страна . столица)}, и возвращают по стране столицу, а по столице --- страну

\subsection*{Решение}
\begin{lstlisting}
(defun get-capital (table country)
	(cdr (assoc country table)))

(defun get-country (table capital)
	(car (rassoc capital table)))
\end{lstlisting}


\section*{Задание №4}
\subsection*{Постановка задачи}
Напишите функцию \texttt{swap-first-last}, которая переставляет в списке аргументе первый и последний элементы

\subsection*{Решение}
\begin{lstlisting}
(defun swap-first-last (lst)
	(append
		(append
			(cons (car (reverse lst)) Nil) ;; last
			(reverse (cdr (reverse lst)))) ;; tail without last
		(cons (car lst) Nil))) ;; head
\end{lstlisting}

\section*{Задание №5}
\subsection*{Постановка задачи}
Напишите функцию \texttt{swap-two-ellement}, которая переставляет в списке-аргументе два указанных своими порядковыми номерами элемента в этом списке

\subsection*{Решение}
\begin{lstlisting}
(defun swap-two-elements (lst ind1 ind2)
	(rotatef (nth ind1 lst) (nth ind2 lst)) lst)
\end{lstlisting}

\section*{Задание №6}
\subsection*{Постановка задачи}
Напишите две функции, \texttt{swap-to-left} и \texttt{swap-to-right}, которые производят круговую перестановку в списке-аргументе влево и вправо, соответственно

\subsection*{Решение}
\begin{lstlisting}
(defun swap-to-left (lst)
	(append (cdr lst) (cons (car lst) Nil)))

(defun swap-to-right (lst)
	(append
		(cons (car (reverse lst)) Nil)
		(reverse (cdr (reverse lst)))))
\end{lstlisting}


\section*{Контрольные вопросы}
\textbf{Вопрос 1.} Способы определения функций\\
\indent\textbf{Ответ.} Существует два способа определений функций:

\begin{itemize}
	\item через \texttt{defun};
	\item через \texttt{lambda}.
\end{itemize}

Пример \texttt{defun}:
\begin{lstlisting}
(defun func-name (args-list) function-body)
(defun get-cube(y) (* y y y))
(get-cube y)
\end{lstlisting}

Пример \texttt{lambda}:
\begin{lstlisting}
(lambda (args-list) function-body)
((lambda (x) (* x x)) 2)
\end{lstlisting} ""\newline

\indent \textbf{Вопрос 2.} Варианты и методы модификации списков\\
\indent\textbf{Ответ. }\textbf{Не разрушающие структуру списка функции.} Данные функции не меняют сам объект-аргумент, а создают копию. К таким функциям относятся: 
\texttt{append, reverse, last, nth, nthcdr, length, remove, subst} и прочие.\\

\indent \textbf{Структуроразрушающие функции.} Данные функции меняют сам объект-аргумент, невозможно вернуться к исходному списку. Чаще всего такие функции начинаются с префикса \texttt{n-}. К такми функция относятся: \texttt{nreverse, nconc, nsubst} и прочие.\\

\textbf{Вопрос 3.} Отличие в работе функций \texttt{cons}, \texttt{list}, \texttt{append} и в их результате\\
\indent\textbf{Ответ. }Функция \texttt{cons} --- чисто математическая, конструирует списковую ячейку, которая может вовсе и не быть списком. Является списком только в том случае, если вторым аргументом передан список.\\

Функция \texttt{list} --- форма, принимает произвольное количество аргументов и конструирует из них список. Результат --- всегда список. При нуле аргументов возвращает пустой список.\\

Функция \texttt{append} --- форма, принимает на вход произвольное количество аргументов и для всех аргументов, кроме последнего, создает копию, ссылая при этом последний элемент каждого списка-аргумента на первый элемент следующего по порядку списка-аргумента. Копирование для последнего не делается в целях эффективности.\\

	
\bibliographystyle{utf8gost705u}  % стилевой файл для оформления по ГОСТу
	
\bibliography{51-biblio}          % имя библиографической базы (bib-файла)
	
	
\end{document}
