\documentclass[12pt]{report}
\usepackage[utf8]{inputenc}
\usepackage[russian]{babel}
%\usepackage[14pt]{extsizes}
\usepackage{listings}
\usepackage{graphicx}
\usepackage{amsmath,amsfonts,amssymb,amsthm,mathtools} 
\usepackage{pgfplots}
\usepackage{filecontents}
\usepackage{float}
\usepackage{indentfirst}
\usepackage{eucal}
\usepackage{enumitem}
%s\documentclass[openany]{book}
\frenchspacing

\usepackage{indentfirst} % Красная строка

\usetikzlibrary{datavisualization}
\usetikzlibrary{datavisualization.formats.functions}

\usepackage{amsmath}


% Для листинга кода:
\lstset{ %
	language=c,                 % выбор языка для подсветки (здесь это С)
	basicstyle=\small\sffamily, % размер и начертание шрифта для подсветки кода
	numbers=left,               % где поставить нумерацию строк (слева\справа)
	numberstyle=\tiny,           % размер шрифта для номеров строк
	stepnumber=1,                   % размер шага между двумя номерами строк
	numbersep=5pt,                % как далеко отстоят номера строк от подсвечиваемого кода
	showspaces=false,            % показывать или нет пробелы специальными отступами
	showstringspaces=false,      % показывать или нет пробелы в строках
	showtabs=false,             % показывать или нет табуляцию в строках
	frame=single,              % рисовать рамку вокруг кода
	tabsize=2,                 % размер табуляции по умолчанию равен 2 пробелам
	captionpos=t,              % позиция заголовка вверху [t] или внизу [b] 
	breaklines=true,           % автоматически переносить строки (да\нет)
	breakatwhitespace=false, % переносить строки только если есть пробел
	escapeinside={\#*}{*)}   % если нужно добавить комментарии в коде
}


\usepackage[left=2cm,right=2cm, top=2cm,bottom=2cm,bindingoffset=0cm]{geometry}
% Для измененных титулов глав:
\usepackage{titlesec, blindtext, color} % подключаем нужные пакеты
\definecolor{gray75}{gray}{0.75} % определяем цвет
\newcommand{\hsp}{\hspace{20pt}} % длина линии в 20pt
% titleformat определяет стиль
\titleformat{\chapter}[hang]{\Huge\bfseries}{\thechapter\hsp\textcolor{gray75}{|}\hsp}{0pt}{\Huge\bfseries}


% plot
\usepackage{pgfplots}
\usepackage{filecontents}
\usetikzlibrary{datavisualization}
\usetikzlibrary{datavisualization.formats.functions}

\begin{document}
	%\def\chaptername{} % убирает "Глава"
	\thispagestyle{empty}
	\begin{titlepage}
		\noindent \begin{minipage}{0.15\textwidth}
			\includegraphics[width=\linewidth]{img/b_logo}
		\end{minipage}
		\noindent\begin{minipage}{0.9\textwidth}\centering
			\textbf{Министерство науки и высшего образования Российской Федерации}\\
			\textbf{Федеральное государственное бюджетное образовательное учреждение высшего образования}\\
			\textbf{~~~«Московский государственный технический университет имени Н.Э.~Баумана}\\
			\textbf{(национальный исследовательский университет)»}\\
			\textbf{(МГТУ им. Н.Э.~Баумана)}
		\end{minipage}
		
		\noindent\rule{18cm}{3pt}
		\newline\newline
		\noindent ФАКУЛЬТЕТ $\underline{\text{«Информатика и системы управления»}}$ \newline\newline
		\noindent КАФЕДРА $\underline{\text{«Программное обеспечение ЭВМ и информационные технологии»}}$\newline\newline\newline\newline\newline
		
		\begin{center}
			\noindent\begin{minipage}{1.1\textwidth}\centering
				\Large\textbf{  Отчет по лабораторной работе №8}\newline
				\textbf{по дисциплине <<Функциональное и логическое}\newline
				\textbf{~~~программирование>>}\newline\newline
			\end{minipage}
		\end{center}
		
		\noindent\textbf{Тема} $\underline{\text{Использование функционалов~~~~~~~~~~~~~~~~~~~~~~}}$\newline\newline
		\noindent\textbf{Студент} $\underline{\text{Романов А.В.~~~~~~~~~~~~~~~~~~~~~~~~~~~~~~~~~~~~~~~~}}$\newline\newline
		\noindent\textbf{Группа} $\underline{\text{ИУ7-63Б~~~~~~~~~~~~~~~~~~~~~~~~~~~~~~~~~~~~~~~~~~~~~~~~}}$\newline\newline
		\noindent\textbf{Оценка (баллы)} $\underline{\text{~~~~~~~~~~~~~~~~~~~~~~~~~~~~~~~~~~~~~~~~~~~~~~~}}$\newline\newline
		\noindent\textbf{Преподаватель} $\underline{\text{Толпинская Н.Б., Строганов Ю. В.}}$\newline\newline\newline
		
		\begin{center}
			\vfill
			Москва~---~\the\year
			~г.
		\end{center}
	\end{titlepage}
	
	
\section*{Задание 1}
\subsection*{Постановка задачи}
Напишите функцию, которая умножает на заданное число-аргумент все числа из заданного списка-аргумента, когда все элементы списка --- числа и все элементы списка --- любые объекты.

\subsection*{Решение}
\begin{lstlisting}
(defun mult-numbers (acc lst)
	(reduce #'* lst :initial-value acc))

(defun mult-objects (mp lst)
	(reduce
		#'(lambda (acc el)
			(if (numberp el)
				(* acc el)
				acc))
	lst :initial-value mp))
\end{lstlisting}


\section*{Задание №2}
\subsection*{Постановка задачи}
Напишите функцию, \texttt{select-between}, которая из списка-аргумента, содержащего только числа, выбирает только те, которые расположены между двумя указанными границами-аргументами и возвращает их в виде списка (упорядоченного по возрастанию списка чисел)

\subsection*{Решение}
\begin{lstlisting}
(defun rec-add-to-end (lst elem)
	(cond 
		((cdr lst) (rec-add-to-end (cdr lst) elem))
		((listp elem) (setf (cdr lst) elem))
		(t (setf (cdr lst) (cons elem Nil))))
	lst)

(defun add-to-end (lst elem)
	(if (null lst)
		(cons elem Nil)
		(rec-add-to-end lst elem)))

(defun select-between (lst bot top)
	(reduce
		#'(lambda (acc el)
			(if (and (> el bot) (< el top))
				(add-to-end acc el)
		acc))
	lst :initial-value ()))
\end{lstlisting}

\section*{Задание №3}
\subsection*{Постановка задачи}
Что будет результатом \texttt{(mapcar 'вектор '(570-40-8))}?

\subsection*{Решение}
Функции \texttt{вектор} не существует, программа завершится с ошибкой.

\section*{Задание №4}
Напишите функцию, которая уменьшает на 10 все числа из списка аргумента этой функции.

\subsection*{Постановка задачи}
\subsection*{Решение}
\begin{lstlisting}
(defun rec-minus-n-internal (lst n acc)
	(if (car lst)
		(cond ((listp (car lst))
			(add-to-end acc (rec-minus-n (car lst) n ())))
		((numberp (car lst))
			(rec-minus-n (cdr lst) n (add-to-end acc (- (car lst) n))))
		(t
			(add-to-end acc (car lst))))
	acc))

(defun rec-minus-n (lst n)
	(rec-minus-n-internal lst n ()))

(defun minus-n (lst n)
	(mapcar #'(lambda (el)
		(cond 
			((listp el) (minus-n el n))
			((numberp el) (- el n))
			(t el)))
	lst))
\end{lstlisting}


\section*{Задание №5}
Написать функцию, которая возвращает первый аргумент списка-аргумента, который сам является непустым списком.

\subsection*{Постановка задачи}
\subsection*{Решение}
\begin{lstlisting}
(defun first-lst (lst)
	(if (listp (car lst))
		(car (car lst))
		(first-lst (cdr lst))))
\end{lstlisting}
\clearpage

\section*{Задание №6}
Найти сумму числовых элементов смешанного структурированного списка.

\subsection*{Постановка задачи}
\subsection*{Решение}
\begin{lstlisting}
(defun sum-list (lst)
	(reduce
		#'(lambda (acc x)
		(cond 
			((listp x) (+ acc (sum-list x)))
			((numberp x) (+ acc x))
			(t acc)))
	lst :initial-value 0))
\end{lstlisting}


\section*{Контрольные вопросы}
\textbf{Вопрос 1.} Порядок работы и варианты использования функционалов\\

Функционалы --- функции, которые в качестве одного из аргументов используют другую функцию.\\

\indent\textbf{Ответ.} \textbf{Применяющие функционалы}\\

Данные функционалы просто применить переданную в качестве аргумента функцию к переданным в качестве аргументов параметрам.

\begin{enumerate}
	\item \texttt{funcall} -- вызывает функцию-аргумент с остальными аргументами;
	
	Синтаксис: \texttt{(funcall \#'fun arg1 arg2 ... argN)}.
	Пример: \texttt{(funcall \#'* 1 2 3)}
	
	\item \texttt{apply} -- вызывает функцию-аргумент с аргументами из списка, переданного вторым аргументом в \texttt{apply}.
	
	Синтаксис: \texttt{(apply \#'fun arg-lst)}.
	Пример: \texttt{(apply \#'* '(1 2 3))}
\end{enumerate}

\textbf{Отображающие функционалы}\\

Отображения множества аргументов в множество значений позволяют многократно применить функцию. Данные функции берут аргумент, являющийся функцональным объектом и многократно применяет эту фукнцию к элементам переданного в качестве аргумента списка.

\begin{enumerate}
	\item \texttt{mapcar} --- функция \texttt{fun} применяется ко всем первым элементам списков-аргументов, затем ко всем вторым аргументам и так до тех пор, пока не кончатся элементы самого короткого списка. К полученным результатам применения функции применяется функция \texttt{list}, поэтому на выходе функции всегда будет список;
	
	Синтаксис: \texttt{(mapcar \#'fun lst1 lst2 ... lstN)}.
	Пример: \texttt{(mapcar \#'(lambda (x y) (* x y)) '(1 2 3) '(4 5 6)) -> (4 10 18)}
	
	\item \texttt{maplist} --- в качестве аргумента на каждой итерации функция \texttt{fun} получает хвост списка, который использовался на предыдущей итерации (изначально функция получает сам список-аргумент). Если функция принимает несколько аргументов и передано несколько аргументов-списков, то они передаются функции \texttt{fun} в том же порядке, в которым идут в \texttt{maplist};
	
	Синтаксис: \texttt{(maplist \#'fun lst1 lst2 ... lstN)}.
	Пример: \texttt{(maplist \#'(lambda (x y) (+ (car x) (car y))) '(1 2 3 4) '(6 5 4)) -> (list (+ 1 6) (+ 2 5) (+ 2 4))}
	
	\item \texttt{mapcan} --- работает аналогично \texttt{mapcar}, только соединяет результаты функции с помощью функции \texttt{nconc}. Может использоваться как \texttt{filter-map} из некоторых современных языков (например, функция, которая оставляет только четные числа и возводит их в квадрат);
	
	Синтаксис: \texttt{(mapcan \#'fun lst1 lst2 ... lstN)}.
	Пример: \texttt{(mapcan \#'(lambda (x) (and (oddp x) (list (* x x)))) '(1 2 3 4 5 6 7 8 9)) -> (1 9 25 49 81)}
	
	\item \texttt{mapcon} --- работает аналогично \texttt{maplist}, только соединяет результаты функции с помощью функции \texttt{nconc}.
	
	Синтаксис: \texttt{(mapcon \#'fun lst1 lst2 ... lstN)}
	
	\item \texttt{reduce} -- применяет функцию к элементам списка каскадно. Накапливает значение, применяя функцию-аргумент к результату предыдущей итерации и следующему элементу списка (изначально инициализирует результат первым элементом, в случае пустого списка пытается вызвать функцию-аргумент без аргументов и вернуть значение);
	
	Синтаксис: \texttt{(reduce \#'aggregator lst)}.
	Пример: \texttt{(reduce \#'oddp '(1 2 3 4 5 6)) -> (2 4 6)}.

\end{enumerate}

\bibliographystyle{utf8gost705u}  % стилевой файл для оформления по ГОСТу
	
\bibliography{51-biblio}          % имя библиографической базы (bib-файла)
	
	
\end{document}
