\documentclass[12pt]{report}
\usepackage[utf8]{inputenc}
\usepackage[russian]{babel}
%\usepackage[14pt]{extsizes}
\usepackage{listings}
\usepackage{graphicx}
\usepackage{amsmath,amsfonts,amssymb,amsthm,mathtools} 
\usepackage{pgfplots}
\usepackage{filecontents}
\usepackage{float}
\usepackage{indentfirst}
\usepackage{eucal}
\usepackage{enumitem}
%s\documentclass[openany]{book}
\frenchspacing

\usepackage{indentfirst} % Красная строка

\usetikzlibrary{datavisualization}
\usetikzlibrary{datavisualization.formats.functions}

\usepackage{amsmath}


% Для листинга кода:
\lstset{ %
	language=c,                 % выбор языка для подсветки (здесь это С)
	basicstyle=\small\sffamily, % размер и начертание шрифта для подсветки кода
	numbers=left,               % где поставить нумерацию строк (слева\справа)
	numberstyle=\tiny,           % размер шрифта для номеров строк
	stepnumber=1,                   % размер шага между двумя номерами строк
	numbersep=5pt,                % как далеко отстоят номера строк от подсвечиваемого кода
	showspaces=false,            % показывать или нет пробелы специальными отступами
	showstringspaces=false,      % показывать или нет пробелы в строках
	showtabs=false,             % показывать или нет табуляцию в строках
	frame=single,              % рисовать рамку вокруг кода
	tabsize=2,                 % размер табуляции по умолчанию равен 2 пробелам
	captionpos=t,              % позиция заголовка вверху [t] или внизу [b] 
	breaklines=true,           % автоматически переносить строки (да\нет)
	breakatwhitespace=false, % переносить строки только если есть пробел
	escapeinside={\#*}{*)}   % если нужно добавить комментарии в коде
}


\usepackage[left=2cm,right=2cm, top=2cm,bottom=2cm,bindingoffset=0cm]{geometry}
% Для измененных титулов глав:
\usepackage{titlesec, blindtext, color} % подключаем нужные пакеты
\definecolor{gray75}{gray}{0.75} % определяем цвет
\newcommand{\hsp}{\hspace{20pt}} % длина линии в 20pt
% titleformat определяет стиль
\titleformat{\chapter}[hang]{\Huge\bfseries}{\thechapter\hsp\textcolor{gray75}{|}\hsp}{0pt}{\Huge\bfseries}


% plot
\usepackage{pgfplots}
\usepackage{filecontents}
\usetikzlibrary{datavisualization}
\usetikzlibrary{datavisualization.formats.functions}

\begin{document}
	%\def\chaptername{} % убирает "Глава"
	\thispagestyle{empty}
	\begin{titlepage}
		\noindent \begin{minipage}{0.15\textwidth}
			\includegraphics[width=\linewidth]{img/b_logo}
		\end{minipage}
		\noindent\begin{minipage}{0.9\textwidth}\centering
			\textbf{Министерство науки и высшего образования Российской Федерации}\\
			\textbf{Федеральное государственное бюджетное образовательное учреждение высшего образования}\\
			\textbf{~~~«Московский государственный технический университет имени Н.Э.~Баумана}\\
			\textbf{(национальный исследовательский университет)»}\\
			\textbf{(МГТУ им. Н.Э.~Баумана)}
		\end{minipage}
		
		\noindent\rule{18cm}{3pt}
		\newline\newline
		\noindent ФАКУЛЬТЕТ $\underline{\text{«Информатика и системы управления»}}$ \newline\newline
		\noindent КАФЕДРА $\underline{\text{«Программное обеспечение ЭВМ и информационные технологии»}}$\newline\newline\newline\newline\newline
		
		\begin{center}
			\noindent\begin{minipage}{1.1\textwidth}\centering
				\Large\textbf{  Отчет по лабораторной работе №3}\newline
				\textbf{по дисциплине <<Функциональное и логическое}\newline
				\textbf{~~~программирование>>}\newline\newline
			\end{minipage}
		\end{center}
		
		\noindent\textbf{Тема} $\underline{\text{Работа интерпретатора Lisp~~~~~~~~~~~}}$\newline\newline
		\noindent\textbf{Студент} $\underline{\text{Романов А.В.~~~~~~~~~~~~~~~~~~~~~~~~~~}}$\newline\newline
		\noindent\textbf{Группа} $\underline{\text{ИУ7-63Б~~~~~~~~~~~~~~~~~~~~~~~~~~~~~~~~~~}}$\newline\newline
		\noindent\textbf{Оценка (баллы)} $\underline{\text{~~~~~~~~~~~~~~~~~~~~~~~~~~~~~~~~~}}$\newline\newline
		\noindent\textbf{Преподаватель} $\underline{\text{Толпинская Н.Б.~~~~~~~~~~~}}$\newline\newline\newline
		
		\begin{center}
			\vfill
			Москва~---~\the\year
			~г.
		\end{center}
	\end{titlepage}
	
	
	
\section*{Задание 1}
\subsection*{Постановка задачи}

Составить диаграмму вычисления следующих выражений:

\begin{enumerate}
	\item (equal 3 (abs -3))
	\item (equal (+ 1 2) 3)
	\item (equal (* 4 7) 21)
	\item (equal (* 2 3) (+ 7 2))
	\item (equal (- 7 3) (* 3 2)))
	\item (equal (abs (- 2 4)) 3)
\end{enumerate}

\subsection*{Решение}
Решение оформленно на тетрадном листке бумаге. К отчету прилагается.

\section*{Задание 2}
\subsection*{Постановка задачи}

Написать функцию, вычисляющую гипотенузу прямоугольного треугольника по заданным катетам и составить диаграмму ее вычисления. Решение.

\subsection*{Решение}

\begin{lstlisting}[label=second,caption=Решение задания №2, language=lisp]
(defun hypot (x y) (sqrt (+ (* x x) (* y y))))
\end{lstlisting}

\section*{Задание 3}
\subsection*{Постановка задачи}
Написать функцию, вычисляющую объем параллелепипеда по 3-м его сторонам, и составить диаграмму ее вычисления.

\subsection*{Решение}

\begin{lstlisting}[label=third,caption=Решение задания №3, language=lisp]
(defun volume (x y z) (* x y z))
\end{lstlisting}

\section*{Задание 4}
\subsection*{Постановка задачи}
Каковы результаты вычисления следующих выражений? 

\subsection*{Решение}

\begin{lstlisting}[label=4xd,caption=Решение задания №4, language=lisp]
(list `a `b c); THE VARIABLE C IS UNBOUND; (list `a `b `c) -> (ABC)
(cons `a (b c)); THE VARIABLE C IS UNBOUND; (cons `a `(bc)) -> (ABC)
(cons `a `(b c)) -> (A B C)
(caddr (1 2 3 4 5)) -> 3
(cons `a `b `c); INVALID NUMBER OF ARGUMENTS; (cons `a `b) -> (A . B)
(list `a (b c)); THE VARIABLE C IS UNBOUND; (list `a `(b c)) -> (A (BC))
(list a `(b c)); THE VARIABLE A IS UNBOUND; (list `a `(b c)) -> (A (BC))
(list (+ 1 `(length `(1 2 3)))) ; (LENGTH `(1 2 3)) is not of type NUMBER; (list (+1 (length `(123)))) -> 4
\end{lstlisting}

\section*{Задание 5}
\subsection*{Постановка задачи}
Написать функцию \textbf{longer\_then} от двух списков- аргументов, которая возвращает T, если первый аргумент имеет большую длину.

\subsection*{Решение}

\begin{lstlisting}[label=5xd,caption=Решение задания №5, language=lisp]
(defun longer_than (l1 l2) (> (length l1) (length l2)))
\end{lstlisting}

\section*{Задание 6}
\subsection*{Постановка задачи}
Каковы результаты вычисления следующих выражений? 

\subsection*{Решение}

\begin{lstlisting}[label=6xd,caption=Решение задания №6, language=lisp]
(cons 3 (list 56)) -> (356)
(cons 3 `(list 56)) -> (3 LIST 56)
(list 3 `from 8 `gives(-9 3))) -> 3 FROM 9 GIVES 6
(+ (length `(1 foo 2 too)) (car `(21 22 23))) -> 25
(cdr `(cons is short for ans)) -> (IS SHORT FOR ANS)
(car (list one two)); VARIABLE ONE IS UNBOUND; (car (list `one `two)); -> ONE
(cons 3 `(list 5 6)) -> (3 LIST 5 6)
(car (list `one `two)) -> ONE
\end{lstlisting}

\section*{Задание 7}
\subsection*{Постановка задачи}
Дана функция \textbf{(defun mystery (x) (list (second x) (first x)))}. Какие результаты вычисления следующих выражений?

\subsection*{Решение}

\begin{lstlisting}[label=7xd,caption=Решение задания №7, language=lisp]
(mystery `(one two)) -> (TWO ONE)
(mystery `free); The value FREE is not of type LIST; (mystery `(free)) -> (NIL FREE)
(mystery (last `one `two)); The value ONE is not of type LIST when binding LIST; (mystery (last (`one `two))) -> (NIL TWO)
(mystery `one `two); INVALID NUMBER OF ARGUMENTS: 2; (mystery `(one two)) -> (TWO ONE)
\end{lstlisting}

	
\section*{Контрольные вопросы}

\textbf{Вопрос 1.} Базис языка Lisp. \newline
\indent\textbf{Ответ. }
Базис языка представлен:
\begin{itemize}
	\item структурами и атомами;
	\item функциями;
\end{itemize}

Функции, входящие в базис языка:
\begin{itemize}
	\item atom, eq, cons, car, cdr;
	\item cond, quote, lambda, eval, label.
\end{itemize}


\textbf{Вопрос 2.} Классификация функций языка Lisp.
	
\textbf{Ответ.} 
	
\begin{itemize}
	\item чистые (с фиксированным количеством аргументов) математические функции;
	\item рекурсивные функции;
	\item специальные функции – формы (принимают произвольное количество аргументов или по разному обрабатывают аргументы);
	\item псевдофункции (создающие «эффект» – отображающие на экране процесс обработки данных и т.п.);
	\item функции с вариативными значениями, выбирающие одно значение;
	\item функции высших порядков – функционалы (используются для построения синтаксически управляемых программ);
\end{itemize}

\textbf{Вопрос 3.} Синтаксис элементов языка и их представление в памяти.


\textbf{Ответ.}""\newline


Точечные пары ::= (<атом>, <атом>) |

(<атом>, <точечная пара>) |

(<точечная пара>, <атом>) |

(<точечная пара>, <точечная пара>)""\newline

\indent Список ::= <пустой список> | <непустой список>, где

<пустой список> ::= () | Nil,

<непустой список> ::= (<первый элемент>, <хвост>),

<первый элемент> ::= <S-выражение>,

<хвост> ::= <список>""\newline


\indent \textbf{Список} -- частный случай S-выражения. Любая структура (точечная пара или список) заключаются в круглые скобки:


\begin{itemize}

	\item (A . B) -- точечная пара;

	\item $(A)$ -- список из одного элемента;

	\item $Nil$ или $()$ -- пустой список;

	\item (A . (B . (C . (D ()))))) или (A B C D) -- непустой список;

	\item Элементы списка могу являться списками: $((A)(B)(CD))$

\end{itemize}


Любая непустая структура в Lisp, в памяти представленна списковой ячейкой, хранящей два указателя: на голову и хвост.""\newline

\textbf{Вопрос 4.} Функции \textbf{car}, \textbf{cdr}.
	
\textbf{Ответ.} Функции $car$, $cdr$ являются базовыми функциями доступа к
данным. car принимает точечную пару или список в качестве аргумента
и возвращает первый элемент или $Nil$, соответственно. $cdr$ принимает точечную пару или список в качестве аргумента и возвращает все элементы
кроме первого или $Nil$, соответственно.""\newline
	
\textbf{Вопрос 5.} Функции \textbf{list}, \textbf{cons}.
	
\textbf{Ответ.} Функции $list$, $cons$ являются функциями создания списков
($cons$ – базовая, $list$ – нет). $cons$ создает списочную ячейку и устанавливает два указателя на аргументы. $list$ принимает переменное число аргументов и возвращает список, элементы которого – переданные в функцию
аргументы.
	
\bibliographystyle{utf8gost705u}  % стилевой файл для оформления по ГОСТу
	
\bibliography{51-biblio}          % имя библиографической базы (bib-файла)
	
	
\end{document}
