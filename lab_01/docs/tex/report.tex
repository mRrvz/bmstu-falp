\documentclass[12pt]{report}
\usepackage[utf8]{inputenc}
\usepackage[russian]{babel}
%\usepackage[14pt]{extsizes}
\usepackage{listings}
\usepackage{graphicx}
\usepackage{amsmath,amsfonts,amssymb,amsthm,mathtools} 
\usepackage{pgfplots}
\usepackage{filecontents}
\usepackage{float}
\usepackage{indentfirst}
\usepackage{eucal}
\usepackage{enumitem}
%s\documentclass[openany]{book}
\frenchspacing

\usepackage{indentfirst} % Красная строка

\usetikzlibrary{datavisualization}
\usetikzlibrary{datavisualization.formats.functions}

\usepackage{amsmath}


% Для листинга кода:
\lstset{ %
	language=c,                 % выбор языка для подсветки (здесь это С)
	basicstyle=\small\sffamily, % размер и начертание шрифта для подсветки кода
	numbers=left,               % где поставить нумерацию строк (слева\справа)
	numberstyle=\tiny,           % размер шрифта для номеров строк
	stepnumber=1,                   % размер шага между двумя номерами строк
	numbersep=5pt,                % как далеко отстоят номера строк от подсвечиваемого кода
	showspaces=false,            % показывать или нет пробелы специальными отступами
	showstringspaces=false,      % показывать или нет пробелы в строках
	showtabs=false,             % показывать или нет табуляцию в строках
	frame=single,              % рисовать рамку вокруг кода
	tabsize=2,                 % размер табуляции по умолчанию равен 2 пробелам
	captionpos=t,              % позиция заголовка вверху [t] или внизу [b] 
	breaklines=true,           % автоматически переносить строки (да\нет)
	breakatwhitespace=false, % переносить строки только если есть пробел
	escapeinside={\#*}{*)}   % если нужно добавить комментарии в коде
}


\usepackage[left=2cm,right=2cm, top=2cm,bottom=2cm,bindingoffset=0cm]{geometry}
% Для измененных титулов глав:
\usepackage{titlesec, blindtext, color} % подключаем нужные пакеты
\definecolor{gray75}{gray}{0.75} % определяем цвет
\newcommand{\hsp}{\hspace{20pt}} % длина линии в 20pt
% titleformat определяет стиль
\titleformat{\chapter}[hang]{\Huge\bfseries}{\thechapter\hsp\textcolor{gray75}{|}\hsp}{0pt}{\Huge\bfseries}


% plot
\usepackage{pgfplots}
\usepackage{filecontents}
\usetikzlibrary{datavisualization}
\usetikzlibrary{datavisualization.formats.functions}

\begin{document}
	%\def\chaptername{} % убирает "Глава"
	\thispagestyle{empty}
	\begin{titlepage}
		\noindent \begin{minipage}{0.15\textwidth}
			\includegraphics[width=\linewidth]{img/b_logo}
		\end{minipage}
		\noindent\begin{minipage}{0.9\textwidth}\centering
			\textbf{Министерство науки и высшего образования Российской Федерации}\\
			\textbf{Федеральное государственное бюджетное образовательное учреждение высшего образования}\\
			\textbf{~~~«Московский государственный технический университет имени Н.Э.~Баумана}\\
			\textbf{(национальный исследовательский университет)»}\\
			\textbf{(МГТУ им. Н.Э.~Баумана)}
		\end{minipage}
		
		\noindent\rule{18cm}{3pt}
		\newline\newline
		\noindent ФАКУЛЬТЕТ $\underline{\text{«Информатика и системы управления»}}$ \newline\newline
		\noindent КАФЕДРА $\underline{\text{«Программное обеспечение ЭВМ и информационные технологии»}}$\newline\newline\newline\newline\newline
		
		\begin{center}
			\noindent\begin{minipage}{1.1\textwidth}\centering
				\Large\textbf{  Отчет по лабораторной работе №1}\newline
				\textbf{по дисциплине <<Функциональное и логическое}\newline
				\textbf{~~~программирование>>}\newline\newline
			\end{minipage}
		\end{center}
		
		\noindent\textbf{Тема} $\underline{\text{Представление списков в виде списочных ячеек}}$\newline\newline
		\noindent\textbf{Студент} $\underline{\text{Романов А.В.~~~~~~~~~~~~~~~~~~~~~~~~~~~~~~~~~~~~~~~~~~}}$\newline\newline
		\noindent\textbf{Группа} $\underline{\text{ИУ7-63Б~~~~~~~~~~~~~~~~~~~~~~~~~~~~~~~~~~~~~~~~~~~~~~~~~~}}$\newline\newline
		\noindent\textbf{Оценка (баллы)} $\underline{\text{~~~~~~~~~~~~~~~~~~~~~~~~~~~~~~~~~~~~~~~~~~~~~~~~~}}$\newline\newline
		\noindent\textbf{Преподаватель} $\underline{\text{Толпинская Н.Б.~~~~~~~~~~~~~~~~~~~~~~~~~~~~}}$\newline\newline\newline
		
		\begin{center}
			\vfill
			Москва~---~\the\year
			~г.
		\end{center}
	\end{titlepage}
	
	
	
	\chapter*{Задание 1}
	\section*{Постановка задачи}
	
	Постановка задачи. Представить следующие списки в виде списочных ячеек:
	
	
	\begin{enumerate}
		\item ’(open close halph)
		
		\item ’((open1) (close2) (halph3))
		
		\item ’((one) for all (and (me (for you))))
		
		\item ’((TOOL)(call))
		
		\item ’((TOOL1)((call2))((sell)))
		
		\item ’(((TOOL)(call))(sell))
	\end{enumerate}
	
	\section*{Решение}
	Решение оформлено на тетрадном листе бумаге, прилагающемуся к отчету.
	
	\chapter*{Контрольные вопросы}
	\textbf{Вопрос 1.} Перечислить элементы языка Lisp.
	
	\textbf{Ответ.} Элементами языка Lisp являются атомы и точечные пары (структуры). К атомам относится:
	
	\begin{itemize}
		\item символы -- набор литер, начинающихся с буквы.
		\item спец. символы: $\{T, Nil\}$. Используются для обозначения логических констант.
		\item самоопределимые атомы -- натуральные, дробные, вещественные числа, строки (последовательность символов, заключенных в двойные апострофы)
	\end{itemize}
	
	\textbf{Вопрос 2.} Синтаксис элементов языка и их представление в памяти.
	
	\textbf{Ответ.}""\newline
	
	Точечные пары ::= (<атом>, <атом>) |
	
	(<атом>, <точечная пара>) |
	
	(<точечная пара>, <атом>) |
	
	(<точечная пара>, <точечная пара>)
	
	
	""\newline
	\indent Список ::= <пустой список> | <непустой список>, где
	
	<пустой список> ::= () | Nil,
	
	<непустой список> ::= (<первый элемент>, <хвост>),
	
	<первый элемент> ::= <S-выражение>,
	
	<хвост> ::= <список>""\newline
	
	\indent \textbf{Список} -- частный случай S-выражения. Любая структура (точечная пара или список) заключаются в круглые скобки:
	
	\begin{itemize}
		\item (A . B) -- точечная пара;
		\item $(A)$ -- список из одного элемента;
		\item $Nil$ или $()$ -- пустой список;
		\item (A . (B . (C . (D ()))))) или (A B C D) -- непустой список;
		\item Элементы списка могу являться списками: $((A)(B)(CD))$
	\end{itemize}
	
	Любая непустая структура в Lisp, в памяти представленна списковой ячейкой, хранящей два указателя: на голову и хвост.""\newline
	
	\textbf{Вопрос 3.} Как воспринимается символ ’ (апостроф)?
	
	\textbf{Ответ.} Символ ’ эквивалентен функции quote – он блокирует вычисление выражения. Таким образом, выражение воспринимается интерпретатором как данные.""\newline
	
	\textbf{Вопрос 4.} Что такое рекурсия и примеры рекурсии из языка Lisp.
	
	\textbf{Ответ.} Рекурсия – ссылка на описываемый объект в процессе его описания. Списки в Lisp заданы рекурсивно, то есть каждый элемент списка
	является еще одним списком, имеющим непустой или пустой хвост.
	
	\bibliographystyle{utf8gost705u}  % стилевой файл для оформления по ГОСТу
	
	\bibliography{51-biblio}          % имя библиографической базы (bib-файла)
	
	
\end{document}
