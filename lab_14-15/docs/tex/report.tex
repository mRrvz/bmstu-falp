\documentclass[12pt]{report}
\usepackage[utf8]{inputenc}
\usepackage[russian]{babel}
%\usepackage[14pt]{extsizes}
\usepackage{listings}
\usepackage{graphicx}
\usepackage{amsmath,amsfonts,amssymb,amsthm,mathtools} 
\usepackage{pgfplots}
\usepackage{filecontents}
\usepackage{float}
\usepackage{comment}
\usepackage{indentfirst}
\usepackage{eucal}
\usepackage{enumitem}
%s\documentclass[openany]{book}
\frenchspacing

\usepackage{indentfirst} % Красная строка

\usetikzlibrary{datavisualization}
\usetikzlibrary{datavisualization.formats.functions}

\usepackage{amsmath}


% Для листинга кода:
\lstset{ %
	language=c,                 % выбор языка для подсветки (здесь это С)
	basicstyle=\small\sffamily, % размер и начертание шрифта для подсветки кода
	numbers=left,               % где поставить нумерацию строк (слева\справа)
	numberstyle=\tiny,           % размер шрифта для номеров строк
	stepnumber=1,                   % размер шага между двумя номерами строк
	numbersep=5pt,                % как далеко отстоят номера строк от подсвечиваемого кода
	showspaces=false,            % показывать или нет пробелы специальными отступами
	showstringspaces=false,      % показывать или нет пробелы в строках
	showtabs=false,             % показывать или нет табуляцию в строках
	frame=single,              % рисовать рамку вокруг кода
	tabsize=2,                 % размер табуляции по умолчанию равен 2 пробелам
	captionpos=t,              % позиция заголовка вверху [t] или внизу [b] 
	breaklines=true,           % автоматически переносить строки (да\нет)
	breakatwhitespace=false, % переносить строки только если есть пробел
	escapeinside={\#*}{*)}   % если нужно добавить комментарии в коде
}


\usepackage[left=2cm,right=2cm, top=2cm,bottom=2cm,bindingoffset=0cm]{geometry}
% Для измененных титулов глав:
\usepackage{titlesec, blindtext, color} % подключаем нужные пакеты
\definecolor{gray75}{gray}{0.75} % определяем цвет
\newcommand{\hsp}{\hspace{20pt}} % длина линии в 20pt
% titleformat определяет стиль
\titleformat{\chapter}[hang]{\Huge\bfseries}{\thechapter\hsp\textcolor{gray75}{|}\hsp}{0pt}{\Huge\bfseries}


% plot
\usepackage{pgfplots}
\usepackage{filecontents}
\usetikzlibrary{datavisualization}
\usetikzlibrary{datavisualization.formats.functions}

\begin{document}
	%\def\chaptername{} % убирает "Глава"
	\thispagestyle{empty}
	\begin{titlepage}
		\noindent \begin{minipage}{0.15\textwidth}
			\includegraphics[width=\linewidth]{img/b_logo}
		\end{minipage}
		\noindent\begin{minipage}{0.9\textwidth}\centering
			\textbf{Министерство науки и высшего образования Российской Федерации}\\
			\textbf{Федеральное государственное бюджетное образовательное учреждение высшего образования}\\
			\textbf{~~~«Московский государственный технический университет имени Н.Э.~Баумана}\\
			\textbf{(национальный исследовательский университет)»}\\
			\textbf{(МГТУ им. Н.Э.~Баумана)}
		\end{minipage}
		
		\noindent\rule{18cm}{3pt}
		\newline\newline
		\noindent ФАКУЛЬТЕТ $\underline{\text{«Информатика и системы управления»}}$ \newline\newline
		\noindent КАФЕДРА $\underline{\text{«Программное обеспечение ЭВМ и информационные технологии»}}$\newline\newline\newline\newline\newline
		
		\begin{center}
			\noindent\begin{minipage}{1.1\textwidth}\centering
				\Large\textbf{  Отчет по лабораторной работе №14 - 15}\newline
				\textbf{по дисциплине <<Функциональное и логическое}\newline
				\textbf{~~~программирование>>}\newline\newline
			\end{minipage}
		\end{center}
		
		\noindent\textbf{Тема} $\underline{\text{Работа программы на языке Prolog~~~~~~~~~~~~~~}}$\newline\newline
		\noindent\textbf{Студент} $\underline{\text{Романов А.В.~~~~~~~~~~~~~~~~~~~~~~~~~~~~~~~~~~~~~~~}}$\newline\newline
		\noindent\textbf{Группа} $\underline{\text{ИУ7-63Б~~~~~~~~~~~~~~~~~~~~~~~~~~~~~~~~~~~~~~~~~~~~~~~}}$\newline\newline
		\noindent\textbf{Оценка (баллы)} $\underline{\text{~~~~~~~~~~~~~~~~~~~~~~~~~~~~~~~~~~~~~~~~~~~~~~}}$\newline\newline
		\noindent\textbf{Преподаватель} $\underline{\text{Толпинская Н.Б., Строганов Ю. В.}}$\newline\newline\newline
		
		\begin{center}
			\vfill
			Москва~---~\the\year
			~г.
		\end{center}
	\end{titlepage}
	

\chapter*{Лабораторная работа №14}
\section*{Постановка задачи}

Используя  базу знаний, хранящую знания (лаб. 13):
\begin{itemize}
	\item «Телефонный справочник»: Фамилия, №тел, Адрес – структура (Город, Улица, №дома, №кв),
	\item «Автомобили»: Фамилия\_владельца, Марка, Цвет, Стоимость, и др.,
	\item «Вкладчики банков»: Фамилия, Банк, счет, сумма, др.
\end{itemize}

Владелец может иметь несколько телефонов, автомобилей, вкладов (Факты). В разных городах есть однофамильцы, в одном городе – фамилия уникальна.

Используя конъюнктивное правило и простой вопрос, обеспечить возможность поиска:

По Марке и Цвету автомобиля найти Фамилию, Город, Телефон и Банки, в которых владелец автомобиля имеет вклады. Лишней информации не находить и не передавать!!!

Владельцев может быть несколько (не более 3-х), один и ни одного.

\begin{enumerate}
	\item Для каждого из трех вариантов словесно подробно описать порядок формирования ответа (в виде таблицы). При этом, указать – отметить моменты очередного запуска алгоритма унификации и полный результат его работы. Обосновать следующий шаг работы системы. Выписать унификаторы – подстановки. Указать моменты, причины и результат отката, если он есть.
	\item Для случая нескольких владельцев (2-х): 
	приведите примеры (таблицы) работы системы при разных порядках следования в БЗ  процедур, и знаний в них: («Телефонный справочник», «Автомобили», «Вкладчики банков», или: «Автомобили», «Вкладчики банков», «Телефонный справочник»). Сделайте вывод: Одинаковы ли: множество работ и объем работ в разных случаях?
	\item Оформите 2 таблицы, демонстрирующие порядок работы алгоритма унификации вопроса и подходящего заголовка правила (для двух случаев из пункта 2) и укажите результаты его работы: ответ и побочный эффект.
\end{enumerate}

\subsection*{Решение}
\begin{lstlisting}
domains
	surname, phone, city, street, house, apartment = string
	address = address(city, street, house, apartment)
	model, color, cost = string
	bank, sum = string

predicates
	man(surname, phone, address)
	car(surname, model, color, cost)
	deposit(surname, bank, sum)

	name_city_bank_phone_by_model_color(model, color, surname, city, bank, phone)

clauses
	man("Alexey", "89096412389", address("Krasnogorsk", "Lesnaya", "17", "5")).
	man("Vladimir", "890955550987", address("Moscow", "Sovetskaya", "134", "15")).
	man("Mikhail", "8100500321", address("Khimki", "Lesnaya", "27", "501")).
	man("Pavel", "87654329867", address("Moscow", "Tikhaya", "105", "52")).
	man("Alexander", "89096421389", address("Moscow", "Baumanskaya", "170", "1")).
	man("Andrey", "89999999999", address("Ekaterinburg", "Lenina", "140", "21")).

	car("Alexey", "AMG", "Black", "100000").
	car("Alexey", "Volvo", "Red", "50000").
	car("Pavel", "Nissan", "White", "709000").
	car("Andrey", "Nissan", "White", "205000").
	car("Mikhail", "Cadillac", "Black", "1000000").
	car("Dmirty", "Honda", "Red", "100500").

	deposit("Alexey", "Tinkoff", "10000000000").
	deposit("Andrey", "Sber", "100500").
	deposit("Pavel", "Sber", "0").
	deposit("Mikhail", "Alpha", "10").

	name_city_bank_phone_by_model_color(Model, Color, Surname, City, Bank, Phone) :- car(Surname, Model, Color, _), man(Surname, Phone, address(City, _, _, _)), deposit(Surname, Bank, _).

goal
	%name_city_bank_phone_by_model_color("AMG", "Black", Surname, City, Bank, Phone).
	%name_city_bank_phone_by_model_color("Nissan", "White", Surname, City, Bank, Phone).
	%name_city_bank_phone_by_model_color("Honda", "Red", Surname, City, Bank, Phone).
\end{lstlisting}

\chapter*{Лабораторная работа №15}
\section*{Постановка задачи}
Создать базу знаний «Собственники», дополнив базу знаний, хранящую знания (лаб. 13):

\begin{itemize}
	\item <<Телефонный справочник>>: Фамилия, №тел, Адрес – структура (Город, Улица, №дома, №кв),
	\item <<Автомобили>>: Фамилия\_владельца, Марка, Цвет, Стоимость, и др.,
	\item <<Вкладчики банков>>: Фамилия, Банк, счет, сумма, др.,
\end{itemize}

знаниями о дополнительной собственности владельца. Преобразовать знания об автомобиле к форме знаний о собственности.

Вид собственности (кроме автомобиля):

\begin{itemize}
	\item Строение, стоимость и другие его характеристики.
	\item Участок, стоимость и другие его характеристики.
	\item Водный\_транспорт, стоимость и другие его характеристики.
\end{itemize}

Описать и использовать вариантный домен: \textbf{Собственность}. Владелец может иметь, но только один объект каждого вида собственности (это касается и автомобиля), или не иметь некоторых видов собственности. 

Используя конъюнктивное правило и разные формы задания одного вопроса (пояснять для какого №задания – какой вопрос), обеспечить возможность поиска:

\begin{enumerate}
	\item Названий всех объектов собственности заданного субъекта.
	\item Названий и стоимости всех объектов собственности заданного субъекта.
	\item Разработать правило, позволяющее найти суммарную стоимость всех объектов собственности заданного субъекта.
\end{enumerate}

Для 2-го пункта и одной фамилии составить таблицу, отражающую конкретный порядок работы системы, с объяснениями порядка работы и особенностей использования доменов.

\subsection*{Решение}
\begin{lstlisting}
domains
	city, street, phone, surname, name = string
	house, flat = integer
	address = addr(city, street, house, flat)
	mark, color, bank = string
	id, amount, price = integer

	object = building(name, price);
		region(name, price);
		water_transport(mark, color, price);
		car(mark, color, price).

predicates
	phone(surname, phone, address)
	bank_depositor(surname, bank, id, amount)
	owner(surname, object)

	all_objects(surname, name)
	all_objects_price(surname, name, price)

clauses
	man("Pavel", "+798523415232", addr("Moscow", "Bassmannaya", 34, 12)).
	man("Alexey", "+79752345123", addr("Moscow", "Lesnaya", 41, 37)).
	man("Mikhail", "+75012354433", addr("Krasnogorsk", "Lenina", 73, 13)).
	man("Dmirty", "+79752341432", addr("Sochi", "Sovetskata", 14, 88)).

	owner("Mikhail", car("BMW", "Green", 1000)).
	owner("Mikhail", region("Nothung", 0)).
	owner("Mikhail", building("Moscow-city", 100500)).
	owner("Alexey", car("BMW", "green", 1000)).
	owner("Alexey", region("Krasnogorsk", 10000)).
	owner("Alexey", building("Mail.ru Office", 20000)).
	owner("Alexey", water_transport("Yacht", "Red", 10000)).
	owner("Dmitry", car("Cadillac", "Black", 304000)).
	owner("Dmitry", building("BMSTU", 200000)).
	owner("Pavel", car("Mercedes", "White", 30000)).
	owner("Pavel", building("Tree", 10)).
	
	deposit("Mikhail", "Tinkoff", 22, 1000).
	deposit("Dmitry", "Sber", 33, 10000).
	deposit("Dmitry", "Alfa", 44, 20000).
	deposit("Alexey", "Sper", 238, 10).
	deposit("Pavel", "Maza", 1, 10000).
	
	all_objects(Surname, Name) :- owner(Surname, car(Name, _, _)).
	all_objects(Surname, Name) :- owner(Surname, building(Name, _)).
	all_objects(Surname, Name) :- owner(Surname, region(Name, _)).
	all_objects(Surname, Name) :- owner(Surname, water_transport(Name, _, _)).
	
	all_objects_price(Surname, Name, Price) :- owner(Surname, car(Name, _, Price)).
	all_objects_price(Surname, Name, Price) :- owner(Surname, building(Name, Price)).
	all_objects_price(Surname, Name, Price) :- owner(Surname, region(Name, Price)).
	all_objects_price(Surname, Name, Price) :- owner(Surname, water_transport(Name, _, Price)).

goal
	%all_objects("Alexey", Name).
	%all_objects_with_price("Pavel", Name, Price).

\end{lstlisting}

\chapter*{Теоретическая часть}

\section*{1. Что собой представляет программа на языке пролог?}

Программа на Prolog представляет собой набор фактов и правил, обеспечивающих получение заключений на основе этих утверждений. Программа содержит базу знаний и вопрос. База знаний содержит истинные значения, используя которые программа выдает ответ на вопрос. 

Основным элементом языка является терм. База знаний состоит из предложений. Каждое предложение заканчивается точкой. Вопрос состоит только из тела – составного терма (или нескольких составных термов). Вопросы используются для выяснения выполнимости некоторого отношения между описанными в программе объектами. Система рассматривает вопрос как цель, к которой (к истинности которой) надо стремиться. Ответ на вопрос может оказаться логически положительным или отрицательным, в зависимости от того, может ли быть достигнута соответствующая цель.

\section*{2. Какова структура программы на Prolog?}

Программа на Prolog состоит из следующих разделов:

\begin{itemize}
	\item директивы компилятора — зарезервированные символьные константы,
	\item CONSTANTS — раздел описания констант,
	\item DOMAINS — раздел описания доменов,
	\item DATABASE — раздел описания предикатов внутренней базы данных,
	\item PREDICATES — раздел описания предикатов,
	\item CLAUSES — раздел описания предложений базы знаний,
	\item GOAL — раздел описания внутренней цели (вопроса).
	В программе не обязательно должны быть все разделы.
\end{itemize}

\section*{3. Как реализуется программа на Prolog? Как формируются результаты работы программы?}

Ответ на поставленный вопрос система дает в логической форме - «Да» или «Нет». Цель системы состоит в том, чтобы на поставленный вопрос найти возможность, исходя из базы знаний, ответить «Да». Вариантов ответить «Да» на поставленный вопрос может быть несколько. В нашем случае система настроена в режим получения всех возможных вариантов ответа. При поиске ответов на вопрос рассматриваются альтернативные варианты и находятся все возможные решения (методом проб и ошибок) - множества значений переменных, при которых на поставленный вопрос можно ответить - «Да».

Для выполнения логического вывода используется механизм унификации, встроенный в систему.
Унификация – операция, которая позволяет формализовать процесс логического вывода. С практической точки зрения  - это основной вычислительный шаг, с помощью которого происходит:
\begin{itemize}
	\item Двунаправленная передача параметров процедурам,
	\item Неразрушающее присваивание,
	\item Проверка условий (доказательство).
\end{itemize}

В процессе работы система выполняет большое число унификаций.  Попытка "увидеть одинаковость" – сопоставимость двух термов, может завершаться успехом или тупиковой ситуацией (неудачей). В последнем случае включается механизм отката к предыдущему шагу.

\section*{4. Что такое терм?}

Терм - основной элемент языка Prolog. Терм – это:

\begin{enumerate}
	\item Константа: 
	\begin{itemize}
		\item Число (целое, вещественное),
		\item Символьный атом (комбинация символов латинского алфавита, цифр и символа подчеркивания, начинающаяся со строчной буквы),
		\item Строка: последовательность символов, заключенных в кавычки.
	\end{itemize}
	\item Переменная:
	\begin{itemize}
		\item Именованная – обозначается комбинацией символов латинского алфавита, цифр и символа подчеркивания, начинающейся с прописной буквы или символа подчеркивания,
		\item Анонимная  - обозначается символом подчеркивания
	\end{itemize}
	\item Составной терм:
	Это средство организации группы отдельных элементов знаний в единый  объект,  синтаксически представляется: f(t1, t2, …,tm), где f -  функтор (отношение между объектами), t1, t2, …,tm – термы, в том  числе  и составные.
\end{enumerate}

\section*{5. Что такое предикат в матлогике (математике)?}

Предикат в математической логике - это утверждение, высказанное о субъекте. Предикат является функцией со значениями {0, 1} (истина/ложь), определенной на некотором множестве параметров. Предикат называю n-арным, если он определен на n-ой декартовой степени множества М.

\section*{6. Что описывает предикат в Prolog?}

Предикат в Prolog описывает отношение между аргументами процедуры. Процедурой в Prolog является совокупность всех правил, описывающих определенное отношение.

\section*{7. Назовите виды предложений в программе и приведите примеры таких предложений из вашей программы. Какие предложения являются основными, а какие - не основными? Каковы: синтаксис и семантика (формальных смысл) этих предложений (основных и неосновных)?}

В Prolog есть два типа предложений: правила и факты. Правило имеет вид: $A$ :- $B_{1}$,... ,$B_{n}$. 
A называется заголовком правила, а $B_{1}$,...,$B_{n}$ – телом правила. Заголовок содержит некоторое знание, а тело - условие истинности этого знания. Факт является частным случаем правила - в нем отсутствует тело. 

\begin{itemize}
	\item Пример факта из программы: \emph{car("Mikhail"{}, "Cadillac"{}, "Black"{}, "500000"). }
	\item Пример правила из программы: \emph{car\_by\_phone(Phone, Surname, Model, Cost) :- man(Surame, Phone, \_), car(Name, Model, \_, Cost). }
\end{itemize}


Основными называются предложения, не содержащие переменных. Предложения, содержащие переменные называются неосновными.\\

Синтаксис предложения: заголовок(составной терм) :- тело(один или последовательность термов). Предложения используются для формирования базы знаний о некоторой предметной области.

\section*{8. Каковы назначение, виды и особенности использования переменных в программе на Prolog? Какое предложение БЗ сформулировано в более общей - абстрактной форме: содержащее или не содержащее переменных?}

Переменные предназначены для обозначения некоторого неизвестного объекта предметной области. Переменные бывают именованными и анонимными. Именованные переменные уникальны в рамках предложения, а анонимная переменная – любая уникальна. В разных предложениях может использоваться одно имя переменной для обозначения разных объектов.

В ходе выполнения программы выполняется связывание переменных с различными объектами, этот процесс называется конкретизацией. Это относится только к именованным переменным. Анонимные переменные не могут быть связаны со значением.

В более общей форме сформулировано предложение, содержащее переменные, так как заранее неизвестно, каким объектом будет конкретизирована переменная.

\section*{9. Что такое подстановка?}

Пусть дан терм: $А(X_1, X_2,  \dots ,X_n)$.
Подстановка - множество пар, вида: $\{X _ i = t _ i\}$, где $X_i$ –   переменная, а $t_i$ –  терм.

\bibliographystyle{utf8gost705u}  % стилевой файл для оформления по ГОСТу
\bibliography{51-biblio}          % имя библиографической базы (bib-файла)
	
\end{document}
