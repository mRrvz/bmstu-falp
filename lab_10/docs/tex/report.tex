\documentclass[12pt]{report}
\usepackage[utf8]{inputenc}
\usepackage[russian]{babel}
%\usepackage[14pt]{extsizes}
\usepackage{listings}
\usepackage{graphicx}
\usepackage{amsmath,amsfonts,amssymb,amsthm,mathtools} 
\usepackage{pgfplots}
\usepackage{filecontents}
\usepackage{float}
\usepackage{indentfirst}
\usepackage{eucal}
\usepackage{enumitem}
%s\documentclass[openany]{book}
\frenchspacing

\usepackage{indentfirst} % Красная строка

\usetikzlibrary{datavisualization}
\usetikzlibrary{datavisualization.formats.functions}

\usepackage{amsmath}


% Для листинга кода:
\lstset{ %
	language=c,                 % выбор языка для подсветки (здесь это С)
	basicstyle=\small\sffamily, % размер и начертание шрифта для подсветки кода
	numbers=left,               % где поставить нумерацию строк (слева\справа)
	numberstyle=\tiny,           % размер шрифта для номеров строк
	stepnumber=1,                   % размер шага между двумя номерами строк
	numbersep=5pt,                % как далеко отстоят номера строк от подсвечиваемого кода
	showspaces=false,            % показывать или нет пробелы специальными отступами
	showstringspaces=false,      % показывать или нет пробелы в строках
	showtabs=false,             % показывать или нет табуляцию в строках
	frame=single,              % рисовать рамку вокруг кода
	tabsize=2,                 % размер табуляции по умолчанию равен 2 пробелам
	captionpos=t,              % позиция заголовка вверху [t] или внизу [b] 
	breaklines=true,           % автоматически переносить строки (да\нет)
	breakatwhitespace=false, % переносить строки только если есть пробел
	escapeinside={\#*}{*)}   % если нужно добавить комментарии в коде
}


\usepackage[left=2cm,right=2cm, top=2cm,bottom=2cm,bindingoffset=0cm]{geometry}
% Для измененных титулов глав:
\usepackage{titlesec, blindtext, color} % подключаем нужные пакеты
\definecolor{gray75}{gray}{0.75} % определяем цвет
\newcommand{\hsp}{\hspace{20pt}} % длина линии в 20pt
% titleformat определяет стиль
\titleformat{\chapter}[hang]{\Huge\bfseries}{\thechapter\hsp\textcolor{gray75}{|}\hsp}{0pt}{\Huge\bfseries}


% plot
\usepackage{pgfplots}
\usepackage{filecontents}
\usetikzlibrary{datavisualization}
\usetikzlibrary{datavisualization.formats.functions}

\begin{document}
	%\def\chaptername{} % убирает "Глава"
	\thispagestyle{empty}
	\begin{titlepage}
		\noindent \begin{minipage}{0.15\textwidth}
			\includegraphics[width=\linewidth]{img/b_logo}
		\end{minipage}
		\noindent\begin{minipage}{0.9\textwidth}\centering
			\textbf{Министерство науки и высшего образования Российской Федерации}\\
			\textbf{Федеральное государственное бюджетное образовательное учреждение высшего образования}\\
			\textbf{~~~«Московский государственный технический университет имени Н.Э.~Баумана}\\
			\textbf{(национальный исследовательский университет)»}\\
			\textbf{(МГТУ им. Н.Э.~Баумана)}
		\end{minipage}
		
		\noindent\rule{18cm}{3pt}
		\newline\newline
		\noindent ФАКУЛЬТЕТ $\underline{\text{«Информатика и системы управления»}}$ \newline\newline
		\noindent КАФЕДРА $\underline{\text{«Программное обеспечение ЭВМ и информационные технологии»}}$\newline\newline\newline\newline\newline
		
		\begin{center}
			\noindent\begin{minipage}{1.1\textwidth}\centering
				\Large\textbf{  Отчет по лабораторной работе №10}\newline
				\textbf{по дисциплине <<Функциональное и логическое}\newline
				\textbf{~~~программирование>>}\newline\newline
			\end{minipage}
		\end{center}
		
		\noindent\textbf{Тема} $\underline{\text{Вложенная рекурсия и функционалы~~~~~~~~~~~~}}$\newline\newline
		\noindent\textbf{Студент} $\underline{\text{Романов А.В.~~~~~~~~~~~~~~~~~~~~~~~~~~~~~~~~~~~~~~~~}}$\newline\newline
		\noindent\textbf{Группа} $\underline{\text{ИУ7-63Б~~~~~~~~~~~~~~~~~~~~~~~~~~~~~~~~~~~~~~~~~~~~~~~~}}$\newline\newline
		\noindent\textbf{Оценка (баллы)} $\underline{\text{~~~~~~~~~~~~~~~~~~~~~~~~~~~~~~~~~~~~~~~~~~~~~~~}}$\newline\newline
		\noindent\textbf{Преподаватель} $\underline{\text{Толпинская Н.Б., Строганов Ю. В.}}$\newline\newline\newline
		
		\begin{center}
			\vfill
			Москва~---~\the\year
			~г.
		\end{center}
	\end{titlepage}
	
	
\section*{Задание №1}
\subsection*{Постановка задачи}
Написать рекурсивную версию (с именем \texttt{rec-add}) вычисления суммы чисел заданного списка.

\subsection*{Решение}
\begin{lstlisting}
(defun rec-add-internal (lst)
	(if (null lst) acc
		(rec-add-internal (cdr lst) (+ acc (car lst)))))

(defun rec-add (lst)
	(rec-add-internal lst 0))
\end{lstlisting}

\section*{Задание №2}
\subsection*{Постановка задачи}
Написать рекурсивную функцию с именем \texttt{rec-nth} функции \texttt{nth}.

\subsection*{Решение}
\begin{lstlisting}
(defun rec-nth (n lst)
	(if (zerop n)
		(car lst)
		(rec-nth (- n 1) (cdr lst))))
\end{lstlisting}

\section*{Задание №3}
\subsection*{Постановка задачи}
Написать рекурсивную функцию \texttt{alloddr}, которая возвращает \texttt{t}, когда все элементы списка нечётные.

\subsection*{Решение}
\begin{lstlisting}
(defun alloddr (lst)
	(if (null lst) T
		(and (oddp (car lst)) (alloddr (cdr lst)))))
\end{lstlisting}

\section*{Задание №4}
\subsection*{Постановка задачи}
Написать рекурсивную функцию, относящуюся к хвостовой рекурсии с одним тестом завершения, которая возвращает последний элемент списка-аргумента.

\subsection*{Решение}
\begin{lstlisting}
(defun rec-last (lst)
	(if (null (cdr lst))
		(car lst)
		(rec-last (cdr lst))))
\end{lstlisting}

\section*{Задание №5}
\subsection*{Постановка задачи}
Написать рекурсивную функцию, относящуюся к дополняемой рекурсии с одним тестом завершения, которая вычисляет сумму всех чисел от 0 до \textit{n}-ого аргумента функции.

\subsection*{Решение}
\begin{lstlisting}
(defun rec-sum-n (n lst)
	(if (or (zerop n) (null lst)) 0
		(+ (car lst) (sum-n (- n 1) (cdr lst)))))
\end{lstlisting}

\section*{Задание №6}
\subsection*{Постановка задачи}
Написать рекурсию, которая возвращает последнее нечетное число из числового списка, возможно создавая некоторые вспомогательные функции.

\clearpage
\subsection*{Решение}
\begin{lstlisting}
(defun rec-last-odd-internal (lst curr)
	(if (null lst) cur
		(if (oddp (car lst))
			(rec-last-odd-internal (cdr lst) (car lst))
			(rec-last-odd-internal (cdr lst) cur))))

(defun rec-last-odd (lst)
	(rec-last-odd-internal lst ()))
\end{lstlisting}

\section*{Задание №7}
\subsection*{Постановка задачи}
Используя \texttt{cons}-дополняемую рекурсию с одним тестом завершения, написать функцию, которая получает как аргумент список чисел, а возвращает список квадратов этих чисел в том же порядке.

\subsection*{Решение}
\begin{lstlisting}
(defun cons-square (lst)
	(and lst (cons (* (car lst) (car lst))
		(cons-square (cdr lst)))))
\end{lstlisting}

\section*{Задание №8}
\subsection*{Постановка задачи}
Написать функцию с именем \texttt{select-odd}, которая из заданного списка выбирает все нечетные числа.

\subsection*{Решение}
\begin{lstlisting}
(defun select-odd (lst)
	(reduce #'(lambda (acc x)
		(if (oddp x)
			(append acc (cons x Nil))
			acc))
	lst :initial-value ()))

(defun select-sum-odd (lst)
	(reduce #'(lambda (acc x)
		(if (oddp x)
			(+ acc x)
			acc))
	lst))
\end{lstlisting}

\section*{Задание №9}
\subsection*{Постановка задачи}
Создать и обработать смешанный структурированный список с информацией:
\begin{itemize}
	\item ФИО;
	\item зарплата;
	\item возраст;
	\item категория (квалификация).
\end{itemize}

Изменить зарплату в зависимости от заданного условия, и подсчитать суммарную зарплату. Использовать композиции функций.

\subsection*{Решение}
\begin{lstlisting}
(setf people (list
	(list
		(cons 'FIO "Sergey Kononenko")
		(cons 'Salary 30000)
		(cons 'Age 21)
		(cons 'Category "Tarantool Presale"))
	(list
		(cons 'FIO "Pavel Perestoronin")
		(cons 'Salary 100500)
		(cons 'Age 20)
		(cons 'Category "Qoolo developer"))
	(list
		(cons 'FIO "Mikhail Nitenko")
		(cons 'Salary 3000)
		(cons 'Age 32)
		(cons 'Category "Superflex C++ developer"))
	(list
		(cons 'FIO "Dmitry Yakuba")
		(cons 'Salary 500)
		(cons 'Age 58)
		(cons 'Category "Kotlin)"))
	)
)
\end{lstlisting}

\begin{lstlisting}
(defun sum-salaries (lst)
	(reduce #'(lambda (acc x)
		(+ acc (cdr (assoc 'Salary x))))
	lst :initial-value 0))

(defun get-value (table key)
	(cdr (assoc key table)))

(defun change-salaries-internal (salary-func curr)
	(list
		(cons 'FIO (get-value curr 'FIO))
		(cons 'Salary (funcall salary-func (get-value curr 'Salary)))
		(cons 'Age (get-value curr 'Age))
		(cons 'Category (get-value curr 'Category))))

(defun change-salaries (lst changep salary-func)
	(mapcar #'(lambda (x)
		(if (funcall changep x)
		(change-salaries-internal salary-func x)
		x))
	lst))

(change-salaries people
				 #'(lambda (x) (< (cdr (assoc 'Age x)) 30)) ;; predicate
				 #'(lambda (x) (+ x 15))) ;; change func
\end{lstlisting}

\bibliographystyle{utf8gost705u}  % стилевой файл для оформления по ГОСТу
	
\bibliography{51-biblio}          % имя библиографической базы (bib-файла)
	
	
\end{document}
